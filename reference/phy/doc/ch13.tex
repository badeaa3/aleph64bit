\chapter{\label{sec-OAR}ALPHA Utility Routines: Printing, Writing
Events, Timing, etc.}
\par
\section{\label{sec-QMT}Program termination}
\par
\fbox{CALL QMTERM ('any message')}
\par
\par
Can be called from anywhere.
\begin{description}\item[\bf{Calls}]QUTERM, QUTHIS, QWMESS.\end{description}
\par
{\bf Input argument:}
\begin{indentlist}{ 3.50cm}{ 3.75cm}
\indentitem{'any message'}
character string,
The message will be printed and should contain the
reason for the program termination.
\end{indentlist}
\section{\label{sec-QWR}Write the current event on the output file}
\par
\fbox{CALL QWRITE}
\par
\par
The file name is specified on the FILO card (see
\ref{sec-DCFILO}). This routine can be called from user routines;
it is
called automatically from QMEVNT
if the COPY option (\ref{sec-DCCOPY}) is selected.
If QWRITE is called more than once for the same event, the event will
be written only once.
 
\section{\label{sec-QWCLAS}Set classification word written to event
directory}
\par
\fbox{CALL QWCLAS (IBIT)}
\par
\par
\begin{indentlist}{ 2.50cm}{ 2.75cm}
\indentitem{Input argument:}
\indentitem{IBIT}
Turn on bit IBIT in classification word. IBIT = 1 $-$ 30.
\end{indentlist}
QWCLAS has to
be called once for each bit which is to be set; i.e.,
if three bits are to be set, QCLASW has to be
called three times.
Note that a call to QWCLAS simply turns on a single
bit while leaving other bits unchanged.  The intial classification
word is the one read from the input file;  therefore,
the classification word must be zeroed before storing your
own values:
\begin{verbatim}
       CALL QWCLAS(0).
\end{verbatim}
If QWCLAS is not called, the classification
word will be set equal to that on the input file.
\par
\section{\label{sec-OARTI}Timing}
\par
\subsection{\label{sec-OARPT}Print job time consumption}
\par
\fbox{CALL QWTIME}
\par
\par
Called automatically from QMTERM.
 
\subsection{\label{sec-OARTE}Measure time consumption of part of
program}
\par
\fbox{CALL QTIMED(INUM)}
\par
\par
This subroutine measures the time between two subsequent calls
to QTIMED. 
Time statistics can be kept for up to 9 different
subroutine calls (INUM = 1 $-$ 9).
The time consumption
summary is printed automatically during job termination.
The summary includes the number of calls and the total time / call.
The first CALL QTIMED sets the start time. The time consumption for
QTIMED itself (0.25 msec on CERNVM) is not subtracted in the time
summary.
(The CERNLIB routines TIMED/TIMAD should not be used with QTIMED.)
 
{\bf Example:}
\begin{verbatim}
    CALL QTIMED(1)
      CODE A
    CALL QTIMED(2)
      CODE B
    CALL QTIMED(3)
 
  Results:
QTIMED  Ncalls  total_time   time/call   %
   1      499      35          0.07     92.2
   2      500       1          0.002     2.6
   3      500       2          0.004     5.2
\end{verbatim}
In this example, time 2 gives the time consumption of `CODE A', time 3
gives the time consumption of `CODE B', and time 1 gives the time consumption
between CALL QTIMED(3) and CALL QTIMED(1).
\section{\label{sec-OARPR}Print routines}
\par
The routines described in this section are used to print information
about events or to print messages.
Some of the routines have the subroutine argument {\bf 'option'}.
 
{\bf 'option'} is composed of one or
several characters. Each character has a special meaning:
\begin{indentlist}{ 1.75cm}{ 2.00cm}
\indentitem{'H'}
print a header line. Without this option, you will get a sequence
of numbers without any description. With this option, an extra
line containing the mnemonic symbols for the numbers given
underneath is printed.
\indentitem{'0'}print an empty line and the header line.
\indentitem{'1'}start at a new page and print the header line.
\indentitem{' `}blank space = no option
\end{indentlist}
More options can be given for specific subroutines.
 
\subsection{\label{sec-OARPM}Print a message}
\par
\fbox{CALL QWMESS ('any message')}
\par
\begin{indentlist}{ 3.50cm}{ 3.75cm}
\indentitem{Input argument:}
\indentitem{'any message'}
(character string or character variable)
If the 1st character of `any message' is `0' or `1',
it is taken as carriage control character ('0': empty
line; `1': new page).
If it is not `0' nor `1', it is taken as part of the message.
\end{indentlist}
\subsection{\label{sec-OARPE}Print a message plus run, event number}
\par
\fbox{CALL QWMESE ('any message')}
\par
 
\subsection{\label{sec-QWE}Print full event summary (many pages)}
\par
\fbox{CALL QWEVNT}
\par
\begin{description}\item[\bf{Calls}]QWHEAD, QWSEC, QWTREE\end{description}
\subsection{\label{sec-QWH}Print event header (one line)}
\par
\fbox{CALL QWHEAD ('option', `any text')}
\par
{\bf Input arguments:}
\begin{indentlist}{ 2.50cm}{ 2.75cm}
\indentitem{'option'}one of `H', `0', or `1' (see \ref{sec-OARPR})
\indentitem{'any text'}message; may be blank space: ` '
\indentitem{Output}
 
see printer output of QWHEAD called with option `H'.
Here, as in many other print routines, it's a matter of taste
which data are important enough to be printed, and comments
are welcome. For better readability, the output should always
fit onto one printer line.
\end{indentlist}
\subsection{\label{sec-QWHF}Print full event header (many lines)}
\par
\fbox{CALL QWHFUL ('option', `any text')}
\par
\par
Subroutine arguments are the same as for QWHEAD.
 
\subsection{\label{sec-QWTK}Print information for ``track''}
\par
\fbox{CALL QWITK (ITK, `option')}
\par
\par
\begin{indentlist}{ 2.50cm}{ 2.75cm}
\indentitem{Input arguments:}
\indentitem{ITK}
ALPHA track number.
\indentitem{'option'}
one of `H', `0', or `1' (see \ref{sec-OARPR}).
`L': Do not print locked tracks.
\indentitem{Output}see printer output when called with option `H'.
 
Meaning of column ``det. data'':
\begin{indentlist}{ 1.25cm}{ 1.50cm}
\indentitem{F }general track fit data are available
\indentitem{T }dE/dx data are available
\indentitem{H }HCAL data are available
\indentitem{M }muon chamber data are available
\indentitem{E }Ecal data are available
\indentitem{H }Hcal data are available
\indentitem{... rightmost characters:}
\indentitem{C }object is associated to one or several charged tracks
\indentitem{E }object is associated to one or several Ecal objects
\indentitem{H }object is associated to one or several Hcal objects
\end{indentlist}
\end{indentlist}
\subsection{\label{sec-QWV}Print information for vertex}
\par
\fbox{CALL QWIVX (IVX, `option')}
\par
\begin{indentlist}{ 2.50cm}{ 2.75cm}
\indentitem{Input arguments:}
\indentitem{IVX}ALPHA vertex number.
\indentitem{'option'}one of `H', `0', or `1' (see \ref{sec-OARPR})
\indentitem{Output}see printer output when called with option `H'.
 
\end{indentlist}
\subsection{\label{sec-QWS}Print summary for categories of
tracks or vertices}
\par
\fbox{CALL QWSEC (ISEC, `option')}
\par
\begin{description}\item[\bf{Calls}]QWITK, QWIVX\end{description}
\begin{indentlist}{ 2.50cm}{ 2.75cm}
\indentitem{Input arguments:}
\indentitem{ISEC}
section number = section in QVEC and QVRT:
\begin{indentlist}{ 2.00cm}{ 2.25cm}
\indentitem{KSOVT}Overlap objects
\indentitem{KSCHT}Charged tracks
\indentitem{KSIST}Isolated = neutral cal objects
\indentitem{KSAST}Cal objects associated to charged tracks
\indentitem{KSV0T}Neutral tracks pointing to reconstructed vertices
\indentitem{KSDCT}Tracks outgoing from reconstructed vertices
\indentitem{KSEFT}Energy flow objects
\indentitem{KSNET}Neutral objects from PCPA
\indentitem{KSGAT}Photons from GAMPEC
\indentitem{KSJET}Jets from energy flow objects
\indentitem{KSMCT}MC particles
\indentitem{KSREV}Reconstructed vertices
\indentitem{KSMCV}MC vertices.
\end{indentlist}
\indentitem{'option'}
one of `0', or `1' (see \ref{sec-OARPR}).
'L': Do not print locked tracks.
\end{indentlist}
\subsection{\label{sec-QWTR}Print decay tree of track ITK.}
\par
\fbox{CALL QWTREE (ITK, `option')}
\par
\begin{indentlist}{ 2.50cm}{ 2.75cm}
\indentitem{Input arguments:}
\indentitem{ITK}
Track / particle number.
\indentitem{'option'}one of `H', `0', or `1' (see \ref{sec-OARPR})
\indentitem{Output:}
 
Similar to CALL QWITK.
\end{indentlist}

\subsection{\label{sec-TXTR}Create \LaTeX   source of the decay tree of track ITK.}
\par
\fbox{CALL TXTREE (MOTHER,option)}
\par
\begin{indentlist}{ 2.50cm}{ 2.75cm}
\indentitem{Input arguments:}
\indentitem{MOTHER}
ALPHA Track number where the tree will start.

\indentitem{option}if option='TRACK\#'
 the ALPHA track number will be displayed as well as the particle names;
 otherwise only the particle code is displayed.

\indentitem{Output:}

The output consists of a file with name txtree.tex, in the user's working
directory on UNIX and on the SCWEEK area on AXAL.

\end{indentlist}

Example of use of this routine :
\begin{verbatim}

C Find the ALPHA track number of a J/PSI :
      NPSI=KPDIR('JPSI',KMONTE) 
C  Find the mother of this J/Psi :
      MOTH=KMOTH(NPSI,1)
C Produce the tree with labels only :
      CALL TXTREE(MOTH,' ')
C Produce the tree with labels and track numbers :
      CALL TXTREE(MOTH,'TRACK#')
\end{verbatim}


\chapter{\label{sec-cextalp}C- and $\mathrm{C^{++}}$ Extended ALPHA}
\par
In this appendix, a brief introduction to the use of
ALPHA from a C or $\mathrm{C^{++}}$ program on UNIX machines is given.

\par  
\section{\label{sec-ALPHAPL} $\mathrm{ALPHA^{++}}$:}


The $\mathrm{ALPHA^{++}}$ project has been started in 1998 with the following
goals:

\begin{itemize}
\item convert the ALEPH data (preferably LEP2 data) from the BOS bank style into
persistent objects and write them to an Objectivity/Db database. 

\item
 rewrite a miniversion of ALPHA in an object oriented
     computing language ($\mathrm{C^{++}}$), based on the Objectivity database. 
          
\end{itemize}

The project is still in progress. 

All relevant informations can be found on the $\mathrm{ALPHA^{++}}$  webpage:
 
         http://alephwww.cern.ch/~disserto/objty

\par

\section{\label{sec-ALPHACE} C-extended ALPHA:}
 

  This C-Extended ALPHA package was set up by 
Bob Jacobsen, who is not any more a member of ALEPH.

No modifications have been done to the preliminary version below since May 1997.
It is not considered as standard ALEPH software, therefore many features may be
unreliable, obsolete or incomplete  now.

\par
\begin{verbatim}
 
"C Extended ALPHA" allows you to use all the facilities of ALPHA
from a C or C++ program.  It is available on UNIX OSF1
 machines.
 
There are three new types defined from integers:  AlphaTrack,
AlphaVertex and AlphaPartNum.  These are used as indices into
the existing ALPHA banks.  Except for two exceptions
defined below, the syntax is as described in the ALPHA manual.
Note that everything is upper case only.
 
An example:
 
  AlphaTrack i,j;
  for (i=KFCHT; i<=KLCHT-1; i++)
    for (j=i+1; j<=KLCHT; j++)
      printf(" %d and %d have mass %f\n",i,j,QM2(i,j));
 
The exceptions are:
1) CQTPN returns a character string, which is complicated
   and error prone in C.  Instead, the C version
   takes the result string as an additional argument:
 
      char name[80];
      CQTPN(KFMCT,name);
 
2) 'Output arguments' are passed as references. For example,
   QDEDXM is declared:
 
     void QDEDXM(AlphaTrack,
            int numHypotheses,
            float hypMass[],
            float hypCharge[],
            float *measuredIonization,
            int *numSamples,
            float *length,
            float expectedIonizationForHypothesis[],
            float expectedIonizationSigma[],
            int *error);
 
   and should be called with:
 
      QDEDXM(i,2,MassArray,ChargeArray,&MeasIon,&NumSamp,
          &Length,ExpectArray,SigmaArray,&error);
 
The function prototypes in the header file serve as the
exact documentation for the argument types.
 
ALPHA expects to call FORTRAN versions of QUEVNT, etc.
The linker sees these as 'quevnt_' (i.e. FORTRAN
routines are moved to lower case and have an appended
underscore).
 
There are three approaches for your code:
1) Provide these routines in FORTRAN, and have them
   call whatever C routines you want to add.
2) Provide C routines called quevnt_, etc, which
   will be directly called from ALPHA.
3) Provide C++ routines of the form
        extern "C" void quevnt_(...)
 
 
The include file "$ALEPH/cxa/include/cxa.h" defines
all the necessary wrapper functions and macros, and should
be included at the start of your program. When
used with C++, some of the wrapper macros are redefined
as inline functions to reduce the number of procedure calls.
 
The library "$ALEPH/cxa/lib/libcxa.a" should be included
before the ALPHA and system libraries.  If you include C++
code, linking is complicated by the need to use the C++ linker
(to resolve static constructors) and still get the correct
FORTRAN system libraries linked.  The following Makefile fragment
will do this:
 
$(NAME): $(OBJFILES) $(LFILES)
   $(LD)  -o $(NAME) $(OBJFILES) $(LFILES) $(LDFLAGS) $(LIBS) $(FORLIBS))
 
where the symbols LD and FORLIBS have previously been set
to machine specific values, perhaps by lines in the .cshrc
of the form:
 
    switch (${OS})
         case OSF1:
                setenv CCC cxx
                setenv FORLIBS '-lfor -lUfor -lots -lutil -lm -lFutil'
                breaksw
         case IRIX5:
         default:
                setenv CCC CC
                setenv FORLIBS '/usr/lib/libftn.so /usr/lib/libc.so -lm '
                breaksw
          case ULTRIX:
                setenv CCC g++
                setenv FORLIBS '-lfor -lUfor -lots -lutil -li -lm'
                breaksw
 
 
Known problems and deficiencies:
 
   No HPUX version yet.
 
   Many more inline functions should be coded in C++.
 
   Some of the C wrappers may be missing (please let
   me know - I can't actually believe some of these
   are being used, but will be happy to add them)
 
   It would make a LOT more sense if the C++ types
   AlphaTrack, AlphaVertex, AlphaPartNum where
   inheritable classes.  Unfortunately, I have not
   found a compatible way to do this - ideas are
   welcome.
 
\end{verbatim}
\newpage

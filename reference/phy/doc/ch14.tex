\chapter{\label{sec-MB}Modifying ALPHA banks}
\par
ALPHA subroutines provide protection against inadvertently
overwriting data read from the
input file. In this section, we describe how to modify the internal
ALPHA
banks (QVEC and QVRT) intentionally.
For ``standard''
operations (creating new tracks,
vector operations, Lorentz transformations, etc.), ALPHA utility
routines are available (see ch. \ref{sec-KI}). The tools described
here
can be used when
standard utilities do not exist.
 
\section{\label{sec-MBU}User track / vertex sections}
\par
The subroutines QSUSTR or QSUSVX may be used to reserve certain
track / vertex numbers for your own exclusive usage; they will never
be
modified by any ALPHA utility routine unless explicitly required.
These routines may be called from the user initialization routine QUINIT.
 
\subsection{\label{sec-MBRST}Reserve user space for tracks}
\par
\fbox{CALL QSUSTR (NUSTR)}
\par
\par
Note that ALPHA does not clear (zero) this user space after each event.
\begin{indentlist}{ 3.50cm}{ 3.75cm}
\indentitem{Input argument}
\indentitem{NUSTR}number of user tracks in bank QVEC
\end{indentlist}
\begin{itemize}
\item The track numbers reserved are 1 ... NUSTR.
\item The first track number used in any ALPHA routine will be NUSTR+1.
\end{itemize}
User track space is allocated only if this routine is called.
\newpage 
\subsection{\label{sec-MBRSV}Reserve user space for vertices}
\par
\fbox{CALL QSUSVX (NUSVX)}
\par
\par
Same as QSUSTR (\ref{sec-MBRST}) :
Replace ``track'' by ``vertex'' and ``QVEC'' by ``QVRT''.
 
Utility routines can be called with user tracks as arguments. For
these
tracks, only the basic attributes (columns 1 to 7) are
modified : QX,QY,QZ,QE,QP,QM,QCH. All other
columns are left unchanged (and NOT set to 0!).
 
\section{\label{sec-MBTV}Modifying track / vertex attributes}
\par
All internal ALPHA banks are standard tabular BOS banks and can be
modified like other banks. For the banks QVEC and QVRT, an additional
possibility is foreseen: these banks are passed as arguments to
subroutine QUEVNT and can be used as ordinary 2$-$dimensional arrays.
\begin{verbatim}
         SUBROUTINE QUEVNT (QT,KT,QV,KV)
         DIMENSION QT(KCQVEC,1), KT(KCQVEC,1), QV(KCQVRT,1), KV(KCQVRT,1)
         ...
         QT(JQVEQP,ITK) = 1.
         CALL ABC (QT,KT,QV,KV)
         END
         ...
         SUBROUTINE ABC (QT,KT,QV,KV)
\end{verbatim}
\begin{indentlist}{ 3.75cm}{ 4.00cm}
\indentitem{Remarks :}
QT and KT (tracks) refer to the same array (integer / real*4)
and actually to the address of the bank QVEC plus a 2$-$word
offset for the bank header (LMHLEN). QV and KV are defined
similarly for bank QVRT (vertices).
 
\indentitem{Dimension :}
Use the mnemonic symbols KCQVEC and KCQVRT
(Fortran parameters defined in QCDE) for the number of
columns. The number of rows can be set to any positive number.
 
\indentitem{QT(JQVEQP,ITK) :}
Row number = ALPHA track number.
column number = attribute. For all attributes, parameters
are available in QCDE. The parameter names follow the usual
convention (see App. \ref{sec-BNKDES}).
``J'' + 3 char. of the bank name + 2 char. attribute description.
\end{indentlist}

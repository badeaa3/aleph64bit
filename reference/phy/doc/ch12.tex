\chapter{\label{sec-OARD}Other ALPHA Physics Routines}
\par
\section{\label{sec-OARDEDX}dE/dx Analysis}
 
\par
\subsection{\label{sec-ARDEDX1}Calculate dE/dx for Track ITK from TPC wires}
\par
\fbox{CALL QDEDX(ITK,NHYP,RMASS,Q,RI,NS,TL,RIEXP,SIGMA,IER)}
\par
This routine is an ALPHA interface to the ALEPHLIB routine TIDHYP.
Note that the user must check the return code IER before trying
to use any of the output arguments $-$ not all charged tracks have
dE/dx information!
\par
{\bf Input arguments:}
\begin{indentlist}{ 3.50cm}{ 3.75cm}
\indentitem{ITK}ALPHA track number of a charged reconstructed track.
\indentitem{NHYP}Number of hypotheses the user wishes to try.
If NHYP=1, then RMASS, Q, RIEXP, and SIGMA may be scalar variables.
\indentitem{RMASS(nhyp)}Array of masses, one for each hypothesis.
\indentitem{Q(nhyp)}Array of charges, one for each hypothesis.
\end{indentlist}
\par
{\bf Output arguments:}
\begin{indentlist}{ 3.50cm}{ 3.75cm}
\indentitem{RI}The measured truncated mean ionization, normalized
such that
RI=1 corresponds to minimum ionizing.
\indentitem{NS}Number of useful wire samples on the track.
\indentitem{TL}Useful length of the track (cm).
\indentitem{RIEXP(nhyp)}Expected ionization for each mass hypothesis,
normalized such that RIEXP=1 corresponds to minimum ionizing.
\indentitem{SIGMA(nhyp)}One standard deviation resolution error for
each
hypothesis.  This is the expected dE/dx
resolution, given NS, TL, RIEXP, and
the momentum resolution.
{\bf Note} that one can calculate a $\chi^2$ with 1 d.o.f. as:
$\chi^2$ =((RI$-$RIEXP)/SIGMA)$^2$.
\indentitem{IER}Error return code=0: successful return.
\begin{itemize}
\item =1:  cannot find the track, or ITK is not a charged KRECO track.
\item =2:  cannot find the measured dE/dx information (bank TEXS).
\item =3:  input KRECO charged track has no dE/dx information.
\item =4:  cannot find the necessary database calibration banks,
TC1X, TC2X, and/or TC3X.
\item =5:  cannot find RUNH or EVEH bank
\item =6:  there is no valid dE/dx calibration for this run
\end{itemize}
\end{indentlist}
 
\par
\subsection{\label{sec-ARDEDXP}Calculate dE/dx for Track ITK from TPC pads}
\par
\fbox{CALL QPADX(ITK,NHYP,RMASS,Q,RI,XNS,TL,RIEXP,SIGMA,IER)}
\par
This routine is an ALPHA interface to the ALEPHLIB routine TPDHYP.
Note that the user must check the return code IER before trying
to use any of the output arguments.
\par
{\bf Input arguments:}
\begin{indentlist}{ 3.50cm}{ 3.75cm}
\indentitem{ITK}ALPHA track number of a charged reconstructed track.
\indentitem{NHYP}Number of hypotheses the user wishes to try.
If NHYP=1, then RMASS, Q, RIEXP, and SIGMA may be scalar variables.
\indentitem{RMASS(nhyp)}Array of masses, one for each hypothesis.
\indentitem{Q(nhyp)}Array of charges, one for each hypothesis.
\end{indentlist}
\par
{\bf Output arguments:}
\begin{indentlist}{ 3.50cm}{ 3.75cm}
\indentitem{RI}Measured ionization, normalized
such that
RI=1 corresponds to minimum ionizing.
\indentitem{XNS}Number of wires on the track (NOT integer!).
\indentitem{TL}Useful length of the track (cm).
\indentitem{RIEXP(nhyp)}Expected ionization for each mass hypothesis,
normalized such that RIEXP=1 corresponds to minimum ionizing.
\indentitem{SIGMA(nhyp)}One standard deviation resolution error for
each
hypothesis.  This is the expected dE/dx
resolution, given XNS, TL, RIEXP, and
the momentum resolution.
{\bf Note} that one can calculate a $\chi^2$ with 1 d.o.f. as:
$\chi^2$ =((RI$-$RIEXP)/SIGMA)$^2$.
\indentitem{IER}Error return code=0: successful return.
\begin{itemize}
\item =1:  cannot find the track, or ITK is not a charged KRECO track.
\item =2:  cannot find the measured dE/dx information
\item =3:  input charged track has no dE/dx information.
\item =4:  cannot find the necessary database calibration banks,
TC1X, TC2X, TP3X and/or TC3X.
\item =5:  cannot find RUNH or EVEH bank
\item =6:  cannot find TCSX or TCGX banks
\end{itemize}
\end{indentlist}

\subsection{\label{sec-OARDDX}Combined dE/dx estimation using both wires and pads}
\par
\fbox{CALL QDDX(ITK,NHYP,RMASS,Q,NS,XNPAD,EXTIM,IER)}
\par

Provided by Tommaso Boccali, 19 April 1999.

\par
 
{\bf Input arguments:}
\begin{indentlist}{ 3.50cm}{ 3.75cm}
\indentitem{ITK}ALPHA track number of a charged reconstructed track.
\indentitem{NHYP}Number of hypotheses the user wishes to try.
If NHYP=1, then RMASS, Q, and EXTIM may be scalar variables.
\indentitem{RMASS(nhyp)}Array of masses, one for each hypothesis.
\indentitem{Q(nhyp)}Array of charges, one for each hypothesis.
\end{indentlist}

{\bf Output arguments:}
\begin{indentlist}{ 3.50cm}{ 3.75cm}
\indentitem{NS}Number of wires on the track, -1 if none.                   
\indentitem{XNPAD}Number of pads on the track (Floating !), -1. if none.
\indentitem{EXTIM(nhyp)} (ri-riexp)/$\sigma$  for each  particle hypothesis.
\indentitem{IER}Error return code\\
=0: an estimator was computed;\\
 =1: no information available.
\end{indentlist}
\par




\subsection{\label{sec-OARDEDM}Modified QDEDX for Monte Carlo (Obsolete ! Kept only for backward-compatibility).}
\par
\fbox{CALL QDEDXM(ITK,NHYP,RMASS,Q,RI,NS,TL,RIEXP,SIGMA,IER)}
\par
To be called only for Monte Carlo datasets produced before December 1993; for more recent datasets one has to use
QDEDX (see the use of the CHTSIM function just below).
\par
This routine serves the same purpose as QDEDX,
but treats Monte Carlo differently.
QDEDX takes the dE/dx from the detailed simulation program TPCSIM.
QDEDXM, however, only takes the number of samples and the track length
from TPCSIM, from which a prediction for the resolution is obtained.
The measured momentum and the Monte Carlo true mass then are used
to predict the mean dE/dx, which is smeared by a gaussian random
number to give the simulated dE/dx.
 
The advantage of this approach is that it is easy to adjust the
parameterization to give agreement with data, whereas to do so
with TPCSIM is nontrivial and would require regeneration of the
Monte Carlo data set.
The disadvantage is that the non$-$gaussian tails (which are
small and arise primarily on the high side, due to unresolved
track overlap) are not simulated.
An option does exist to try to get the best of both worlds:
by calling QMTAIL one can set a parameter to tell QDEDXM to
retain the tail simulated by TPCSIM beyond a specified number
of standard deviations.
The distribution below that number of standard deviations then
is obtained from the gaussian random number generator.
Clearly this solution
is not perfect, since the distributions
generally will not match at the chosen cut value.
 
The arguments to this routine are identical to those of QDEDX.
When QDEDXM is used on Monte Carlo events, error code 6
means that no Monte
Carlo truth information is available.
Note that if QDEDXM is used with real data, it is identical
to QDEDX.
\par
\noindent\fbox{CALL QMTAIL(CUT)}
\par
QMTAIL is an entry point in QDEDXM which can be used to set the
cut value, in standard deviations, beyond which the dE/dx
non$-$gaussian tail produced by TPCSIM is retained.  By default,
CUT is set to 999.

%\newpage

\subsection{\label{sec-OARDEDC}QDEDXM or not QDEDXM ? ( Obsolete ! Kept only for backward-compatibility).}
\par
\fbox{LOGICAL FUNCTION CHTSIM(IVERS)}
\par
Function defined only for MCarlo datasets .
\par
Since the release of TPCSIM version 216 in December 1993 , it is much better
to use QDEDX for MCarlo data as well as for real data . The logical function
CHTSIM is .TRUE. if the Monte$-$Carlo dataset currently read has been produced
with a version of TPCSIM , IVERS , which allows to use QDEDX . If CHTSIM is .FALSE. , one has
to use QDEDXM .
\par
 CHTSIM needs to be called only once for a given Monte-Carlo run .
\par
{\bf Output argument:}
\begin{indentlist}{ 3.50cm}{ 3.75cm}
\indentitem{IVERS}The TPCSIM version number for the current dataset .
\end{indentlist}
\subsection{\label{sec-OARDEDV}Check TPC High Voltage for dE/dx}
\par
\fbox{LOGICAL FUNCTION TCHKHV(KRUN,KEVT,IFLG)}
\par
The function TCHKHV, from the ALEPHLIB, is used to check the
TPC high voltage before using the dE/dx information from the TPC.
It checks the data base bank TDBS to find whether any sectors
were being intentionally
operated at reduced voltage during the run in question.
If so, then only the normal TPC HV bit is checked.
Otherwise, the ``dE/dx'' HV bit is checked.
\par
{\bf Input arguments:}
\begin{indentlist}{ 3.50cm}{ 3.75cm}
\indentitem{KRUN}ALEPH run number.
\indentitem{KEVT}ALEPH event number.
\end{indentlist}
\par
{\bf Output arguments:}
\begin{indentlist}{ 3.50cm}{ 3.75cm}
\indentitem{IFLG}What kind of test was made?
\begin{itemize}
\item =0:  test was made on dE/dx HV bit.
\item =1:  test was made on TPC tracking HV bit.
\item =2:  no test was made (banks not found).  TCHKHV=.FALSE.
\end{itemize}
\indentitem{TCHKHV} = .TRUE. if HV is on; .FALSE. otherwise.
\end{indentlist}
\par
\subsection{\label{sec-OARDEDE}Check Existence of dE/dx Calibration
for Run}
\par
\fbox{LOGICAL FUNCTION TCHKEX(KRUN)}
\par
TCHKEX returns .TRUE. if a valid
dE/dx calibration exists for run KRUN.  If a valid calibration
does not exist (because it has not yet been done or because
the run was bad), then TCHKEX returns .FALSE.
This routine resides in the ALEPHLIB.
\par
\section{\label{sec-OARPAIR}Photon conversions}
\par
\fbox{CALL QPAIRF (I1,I2,DXY,DZ0,DZ2,DTH,RMA,ZMA,XMA,NC1,DIN1,NC2, DIN2,P,IER)}
\par
This routine is an ALPHA interface to the ALEPHLIB routine PAIRFD.
Electrons from photon conversion initially will have parallel
trajectories. This algorithm
finds the point on each helix where the tracks are
parallel in the X$-$Y plane and pass closest together; this point
is called the materialization point.
Note that photon conversions are also found in JULIA, and are available
as V0s (see Sections \ref{sec-AL} and \ref{sec-TVAV0M}).
\par
{\bf Input arguments:}
\begin{indentlist}{ 3.50cm}{ 3.75cm}
\indentitem{I1}ALPHA track number of a charged track.
\indentitem{I2}ALPHA track number of a another charged track.
\end{indentlist}
 
{\bf Output arguments:}
\begin{indentlist}{ 3.50cm}{ 3.75cm}
\indentitem{DXY}distance(cm) in the xy plane between the two tracks
at the closest approach to the materialization point.
\indentitem{DZ0}Distance(cm) in z between the two tracks at the origin.
\indentitem{DZ2}The z separation of the tracks at the
closest approach to
the materialization point.
\indentitem{DTH}the theta difference of the two tracks.
\indentitem{RMA}the rho value at the materialization point.
\indentitem{ZMA}the z value at the materialization point.
\indentitem{XMA}The invariant mass of the tracks at the materialization
point assuming they are both electrons.
\indentitem{NC1,2}Number of coordinates with radius less than RMA
for
track 1,2.  0 if no coordinate information is available or
if there are no such coordinates.
\indentitem{DIN1,2}Radial distance between the coordinate closest
to
the origin and RMA for track 1,2;
variable is 0. if no coordinate
information is available or if there are no such coordinates.
\indentitem{P(3)}Summed momentum of the two tracks at the materialization
point in the order x,y,z.
\indentitem{IER} = 0 if calculation is successful; 1 otherwise.
\end{indentlist}
\par
\section{\label{sec-OARMUID}Muon Identification: QMUIDO}
\par
\fbox{
\parbox{4.6in}{CALL~QMUIDO(ITK,IRUN,IBE,IBT,IM1,IM2,NEXP,NFIR,N10,
\ \ N03,XMULT,RAPP,ANG,ISHAD,SUDNT,IDF,IMCF,IER)}
}
\par
This routine is an ALPHA interface to the ALEPHLIB routine
AMUID.
It collects useful information from the banks
HMAD, MCAD, and MUID. 

For most users, who only look at the identification flag
IDF, this routine is useless since this flag
can be accessed with the statement function KMUIIF (see section \ref{sec-TVAMUID}).
  The only purpose of calling  this routine
is to look in more detail at the muon identification.
\par
  QMUIDO cannot be called when reading a Nano-Dst .
\par
{\bf Input argument:}
\begin{indentlist}{ 3.50cm}{ 3.75cm}
\indentitem{ITK}ALPHA track number of a charged reconstructed track.
\end{indentlist}
\par
{\bf Output arguments:}
\begin{indentlist}{ 3.50cm}{ 3.75cm}
\indentitem{IRUN}No longer filled (needed for backwards compatability)
\indentitem{IBE}Bitmap of the planes EXPECTED to have fired in the HCAL
\indentitem{IBT}Bitmap of the planes which have fired in the HCAL
\indentitem{IM1}Number of associated muon chamber hits in the inner layer
\indentitem{IM2}Number of associated muon chamber hits in the outer layer
\indentitem{NEXP}Number of planes expected to have fired in the HCAL
\indentitem{NFIR}Number of planes fired in the HCAL
\indentitem{N10}Number of planes fired in the last ten expected HCAL planes
\indentitem{N03}Number of planes fired in the last three expected HCAL planes
\indentitem{XMULT}Excess hit multiplicity in the last ten planes on the HCAL
\indentitem{RAPP}Distance between track extrapolation and closest muon chamber hit in
standard deviations (the distribution is only approximately normal)
\indentitem{ANG}    Angle between track extrapolation and closest muon chamber hits in
       standard deviations (the distribution is only approximately normal).
       Only available for tracks with at least one muon chamber hit in each
       layer
\indentitem{ISHAD}Shadowing flag = 0 if track is not shadowed; otherwise it is the JULIA
       track number of the shadowing track.
\indentitem{SUDNT}Sum of HCAL hit to track residuals in the last 10 planes.
\indentitem{IDF}Official muon identification flag.
\begin{itemize}
\item     = 1 if muon flagged only by HCAL
\item     = 2 if muon flagged only by MUON
\item     = 3 if muon flagged by both HCAL and MUON
               3 is the .AND. of 1 and 2
\item     = 10 is one hit in each layer of MUON chambers but failing tight matching criteria
\item     = 11 is good HCAL pattern
\item     = 12 is one and only one MUON hit
\item     = 13 is good HCAL + one and only one muon
\item     = 14 is good HCAL + one hit in each layer
\item     = 15 is one hit in each layer of MUON chambers passing tight matching criteria
\item     = 0 not a muon
\item     = -1 to -15 as above but lost shadowing contest
\end{itemize}
\indentitem{IMCF}     Monte Carlo true source of this track
\begin{itemize}
\item     = 0   ambiguous or data
\item     = 1   primary b
\item     = 2   secondary c
\item     = 3   primary c
\item     = 4   b to tau
\item     = 5   other muon
\item     = 6   non decaying hadron or electron
\item     = 7   decay hadron
\end{itemize}
\indentitem{IER}      No longer filled (needed for backwards compatability)
\end{indentlist}
\par
\section{\label{sec-OARVDET}Utility Routines for VDET Analysis}
\par
\subsection{\label{sec-QVDHIT}Number of VDET hits per layer for track ITK}
\par
\fbox{CALL QVDHIT(ITK,IVHIT)}
\par
{\bf Input argument:}
\begin{indentlist}{ 3.50cm}{ 3.75cm}
\indentitem{ITK}ALPHA track number of a reconstructed charged track.
\end{indentlist}
\par
{\bf Output argument:}
\begin{indentlist}{ 3.50cm}{ 3.75cm}
\indentitem{IVHIT}
\begin{itemize}
\item{IVHIT(1)} Number of VDET hits in $r-\phi$ on inner layer
\item{IVHIT(2)} Number of VDET hits in $r-\phi$ on outer layer
\item{IVHIT(3)} Number of VDET hits in $z$ on inner layer
\item{IVHIT(4)} Number of VDET hits in $z$ on outer layer
\end{itemize}
\end{indentlist}
\par
\subsection{\label{sec-XVDEOK}VDET HV status}
\par
\fbox{LOGICAL FUNCTION XVDEOK(dummy)}
\par
A detailed description of this function can be found in the report ALEPH 91$-$161 ( PHYSIC 91$-$137 ) .
 
The function XVDEOK
returns .TRUE. if the VDET HV is on for the current event and .FALSE. otherwise.
XVDEOK uses the HV bits and also calls KVGOOD (see below).
\par
If you want to use tracks with high quality VDET data,
first, check that the HV is on and second, check that the tracks in which
you are interested have VDET hits (see QVDHIT above).
\par
 XVDEOK cannot work when one is reading a Nano-Dst.
\par
\subsection{\label{sec-KVGOOD}VDET Readout Status}
\par
\noindent\fbox{INTEGER FUNCTION KVGOOD(dummy)}
\par
\par
From time to time, there are problems with the VDET
readout which cause VDET information to be read out when the HV is off.
The VDET hits read out during these periods are just noise and can
distort tracks fitted with the VDET.
The function KVGOOD
identifies whether the current event is in one of the bad periods, and if so,
whether or not the HV was on.
\par
\begin{indentlist}{ 3.50cm}{ 3.75cm}
\indentitem{KVGOOD}readout and HV status:
\begin{itemize}
\item $= 0$:  no readout problems, HV is either ON or OFF.
\item $=+1$:  readout problems, HV is ON.
\item $=-1$:  readout problems, HV is OFF.
\end{itemize}
\end{indentlist}
\par
\section{\label{sec-OARQIPB}B-Tagging routine  QIPBTAG}
\par
\subsection{\label{sec-QIPBCA}General considerations}
\par
QIPBTAG is an algorithm for tagging the presence of long-lived particles written by  D. Brown and M. Frank.
\par
It computes the impact parameter of tracks relative to the primary vertex reconstructed
using the QFNDIP subroutine
(\ref{sec-DCQFND} on p.~\pageref{sec-DCQFND}).
 After calibration, the algorithm converts the track signed impact parameter significance
 into a probability that the track came from the primary vertex found by QFNDIP.
  Track probabilities are
 combined to give jet, hemisphere and event probabilities.
\par
Several important modifications have been done to the original QIPBTAG, especially to take into account the new VDET for data from
1996 and after (VDET96 option, J.Carr and M. Thulasidas, June 1997), to
have a consistent b-tag for LEP2 data, and to get from the database bank QIPC the calibration constants which are dependant on the 
period of data taking.

The last version released with ALPHA124.05 (thanks to I. Tomalin, April 1999) recognises automatically
 the type of data being read and 
gives the best possible b-tagging efficiency. See ALPHA124.NEWS for more details.


\par
To use QIPBTAG properly, one must read the complete description of the method and of
the package in the Aleph Note 92$-$135 (August 1992).
\subsection{\label{sec-QIPBCC}Calling the QIPBTAG routine}
\par
\fbox{CALL QIPBTAG(IRET,NTRK,NJET,ITRK,IFRF2,PRTRK,PRJET,PRHEMI,PREVT)}
\par
{\bf Input argument: none}
\par
{\bf Output arguments:}
\begin{indentlist}{ 3.50cm}{ 3.75cm}
\indentitem{IRET}
\begin{itemize}
\item     = 0   good event , analysis performed
\item     = 1   Number of jets less than 2
\item     = 2   Jets outside solid angle cuts
\item     = -2  No beam spot information
\item     = 3,4 No tracks suitable for btag analysis
\item     = -3  Error in interaction point finding
\end{itemize}
\indentitem{NTRK}Number of tracks used to calculate the probabilities (maximum 400)
\indentitem{NJET}Number of jets found (maximum 40)
\indentitem{ITRK}Array (dim 40) of ALPHA track numbers of the NJET jets found,
sorted by energy
\indentitem{IFRF2}Array (dim 400) of row numbers in FRFT 2 of the NTRK tracks
used in the analysis
\indentitem{PRTRK}Array (dim 400) of the probabilities of each track used in the analysis
\indentitem{PRJET}Array (dim 40) of the probabilities of each jet of ITRK
\indentitem{PRHEMI}Array (dim 2) of the probabilities of each hemisphere
\indentitem{PREVT}Probability of the event
\end{indentlist}
\par
All the above arrays must be properly dimensioned in the calling routine.
\par
Additional QIPBTAG informations are available in the ALPHA include file BTAGRAW, which may be found in
\$ALROOT/alphavsn/inc/btagraw.h
\par
\subsection{\label{sec-QIPBCD}Data cards for  QIPBTAG}
\par
{\bf Mandatory data card :}
\begin{indentlist}{ 3.50cm}{ 3.75cm}
\indentitem{QFND}       QFNDIP is needed to find the event interaction point
\end{indentlist}
\par
{\bf Optional data cards :}
\begin{indentlist}{ 3.50cm}{ 3.75cm}
\indentitem{BHIS nnnn}  Books default diagnostic histograms, offset by  nnnn
 
\indentitem{CALB}            Create a private calibration.  The user must also
                        CALL BTAG$\_$FIT('filename') in QWTERM, and the
                        output calibration cards will be written to the file
                        'filename'.
 
 
\indentitem{NUMJ nn}      with nn$>3$, will force QIPBTAG to calculate the b-tagging probability for the
                          first nn jets of the event, ordered by decreasing energy.
                          If this data card is not provided, nn=3 by default.
 
\indentitem{BTRK nn 'name'}     Define a user track type with ALPHA track name
'name'.
                        Up to 3 track types can be defined per job,
                  nn must be $\leq3$. These tracks will never be removed by
                        QIPBTAG selection cuts.  Tracks should be selected or
                        created in QUEVNT and copied to name 'name' before
                        calling QIPBTAG.  If the track is a mother track
                        created by vertexing (see KVFITN), its daughter
                        tracks are excluded from QIPBTAG, and the FRFT 2 track
                        number is set to 20000 + ALPHA track number.
                        If the CALB card is present, user defined track
                        types will be calibrated along with the rest.
 
\indentitem{BNEG}   Use negative impact parameter tracks in the definition
                        of jet, hemisphere, and event probabilities.  This
                        slightly enhances efficiency at the cost of the
                        negative impact parameter control sample.
\indentitem{NOBG}   Forces BNEG option off, even if VDET96 option is ON.   
                    (Of course, must not be used together with BNEG !).
\indentitem{HJET}        Define the hemispheres according to the leading
                        jet direction.  By default, the event is split
                        by the thrust direction (hemi 1 being positive).
                        The hemisphere axis can also be supplied externally
                        for each event as the ALPHA track of name 'THRUST'.
\indentitem{JRES}    Overwrite the default jet definition and corresponding
                        angular resolution parameterization.  User-supplied
                        jets should have ALPHA track name  'QIPBJETS'.

\indentitem{NQIP}    Include VDET 96 modifications even if one is reading LEP 1 data or MCarlo
                     (VDET description of years 1991 to 1995 included).

\indentitem{OQIP}    Do NOT Include VDET 96 modifications even if one is reading LEP 2 data or MCarlo
                     (VDET description of 1996 and after).
                     Strongly recommended if you are reading  LEP1 data reprocessed in 1998.   
\indentitem{TRA2}   Overwrite the default track selection cuts.
                    This card is for use by experts only .
 
\indentitem{QITI}   Calls the QIPBTCPU routine, see section \ref{sec-QIPBTCP}.
\end{indentlist}
\par
\subsection{\label{sec-QIPBCR}Remarks on QIPBTAG}
 QIPBTAG cannot work  on 1989 and 1990 real data or Monte Carlo (no VDET).
\par
 QIPBTAG                              works properly only on real data for which
the VDET is OK (runs given by the SCANBOOK selection VDET OK), and on
events for which XVDEOK = .TRUE. and for which QFNDIP has found a good
interaction point.


\par
To use QIPBTAG only for events with a good interaction point
and VDET OK  you should call it after performing some tests like
in the example below:
\begin{verbatim}
      IF (XVDEOK(DUMM).AND.XGETBP.AND.QVCHIF(KFREV).GT.0.) THEN
         CALL QIPBTAG(.....)
      ENDIF
\end{verbatim}
 
\par
\subsection{\label{sec-QIPBTCP}Auxiliary routine  QIPBTCPU}
\par
If you want to know where QIPBTAG is spending CPU time , you can add a
\begin{verbatim}
         CALL QIPBTCPU('SUMMARY',2)
\end{verbatim}
  at the end of your program .


\par
\section{\label{sec-QSMCALL}Smear 3-d impact parameter in Monte Carlo: QSMEAR}



     The routine QSMEAR may be used AFTER a call to     QIPBTAG.

Its purpose is to
 smear the three dimensional impact parameter
significance and delete extra tracks in the Monte Carlo. It includes
options for doing this work dependently on the polar angle and
momentum of the tracks.

{\bf Calling sequence:}

 
\par
\fbox{
\parbox{4.6in}{CALL~QSMEAR(FIRST,QIPVER,SMBIN,RTBIN,
\ \         NTRK,NJET,IFRF2,PREVT,PRHEMI,PRJET,PRTRK)}
}
\par

{\bf Input Arguments:}
 
\begin{indentlist}{ 3.50cm}{ 3.75cm}

\indentitem{FIRST}     LOGICAL  = Is this the first call to QSMEAR since
                          calling QIPBTAG (or QIPBTAG2) ?

\indentitem{QIPVER}         CHAR*1   = '1' if being used after call to QIPBTAG\\
                        = '2' if being used after call to QIPBTAG2.
\indentitem{SMBIN}         CHAR*1    = '0' no smearing correction at all.\\
                        = 'N' no binning in theta/p for smearing.\\
                        = 'T' binning in theta for smearing.\\
                        = 'P' binning in p for smearing.\\
                        = 'B' binning in both theta/p for smearing.
\indentitem{RTBIN}         CHAR*1    = '0' no track deletion at all.\\
                        = 'N' no binning in theta/p for deletion.\\
                        = 'T' binning in theta for deletion.\\
                        = 'P' binning in p for deletion.\\
                        = 'B' binning in both theta/p for deletion.
\indentitem{NTRK,NJET,..}
         Same arguments of call to QIPBTAG.

\end{indentlist}

\par
{\bf Output Arguments:}  
 
    Same arguments overwritten with smeared values.

           The -ve tag quantities in /BTAGRAW/ are also smeared.

It is possible to smear Monte Carlo from a different year than the
data by including the card    QPYR in the ALPHA card file with a YYMM  
code as an argument.  Example:

QPYR 9701




\par
\section{\label{sec-BMTAG}Invariant mass b tag: QBMTAG}


       The routine QBMTAG can be used in conjunction with QIPBTAG
       to provide extremely pure (99\%) b hemisphere tagging.

      In addition to providing  a btag for each event
       hemisphere, it  also does so for each jet in the event.

\par
\subsection{\label{sec-QIPBMA}General considerations}
\par

    Invariant mass b tag.

     Can be called after QIPBTAG or QVSRCH, to further increase b purity.

     For each event-half QBMTAG loops over tracks in order of decreasing
     inconsistency with primary vertex, and stops when their combined
     invariant mass exceeds 1.8 GeV. (i.e. The mass of a typical charmed
     hadron). It then returns a tag, CLMASS, indicating the probablity
     that the ``last track" used came from primary vertex.

     For a very high purity tag, CLMASS should be combined with the
     QIPTBAG or QVSRCH tags. The optimal linear combination (together
     with suggested cuts) is:
\begin{verbatim} 
         0.7*CLMASS - 0.3*log10(QIPBTAG) > 2.4
     or   0.7*CLMASS + 0.3*QVSRCH > 8.3
\end{verbatim}  

\par
\subsection{\label{sec-QIPBMC}Calling the QBMTAG routine}
  

\par
\fbox{
\parbox{4.6in}{CALL~QBMTAG(METHOD,NTRACK,NJET,TRKJET,PROBTRK,
\ \ FRF2TRK,PVTX,SVTX,DJET,CLMASS,CTSTAR,NSEC,CLMASJ,
\ \ CTSTAJ,NSECJ,LISTTRK,LISTEH,LISTJET)}
}
\par

{\bf INPUT ARGUMENTS:}

\begin{indentlist}{ 3.50cm}{ 3.75cm}
\indentitem{METHOD}
       INTE  : =1 if using QIPBTAG, =2 if using QVSRCH.

\end{indentlist}

  {\bf    Following arguments only needed if METHOD=1\\ 
     (dummy otherwise):}

\begin{indentlist}{ 3.50cm}{ 3.75cm}

\indentitem{NTRACK}
        INTE     : Number of tracks used by QIPBTAG.
\indentitem{NJET}
        INTE       : Number of jets considered by QIPBTAG.
\indentitem{TRKJET(*)}
         INTE  : ALPHA track number of QIPBTAG jets.
\indentitem{PROBTRK(*)}
          REAL : QIPBTAG track probabilities.
\indentitem{FRF2TRK(*)}
   INTE : Row in FRFT bank of QIPBTAG tracks.
\end{indentlist}

{\bf   Following arguments only needed if METHOD=2\\
 (dummy otherwise).}

\begin{indentlist}{ 3.50cm}{ 3.75cm}

\indentitem{PVTX(3)}
        REAL   : Position of primary vertex from QVSRCH.

         Only needed if METHOD=2 (dummy otherwise), same for following arguments.
\indentitem{SVTX(3,2)}                                  
        REAL  : Positions of two secondary vertices in jet
                         coordinate system from QVSRCH.
\indentitem{DJET(3,2)}
        REAL DJET(3,2) : Unitized jet directions from QVSRCH.
  
\end{indentlist}

\par
{\bf OUTPUT ARGUMENTS:\\
 Dimensioned 2 to correspond to two event-halves:}
 
\begin{indentlist}{ 3.50cm}{ 3.75cm}
\indentitem{CLMASS(2)}
        REAL  : For METHOD=1, = -log10(confidence level) for
                     ``last track" to come from primary vertex
                     (c.l. in range 0-1).\\
                     For METHOD=2, = QSQT(CHI**2 (primary) minus
                     CHI**2 (secondary)) for ``last track".
\indentitem{CTSTAR(2)}
        REAL : Decay angle of ``last track" in ``b" rest frame.
\indentitem{NSEC(2)}
        INTE   : Number of tracks used to reach 1.8 GeV mass.
                     (zero if couldn't reach 1.8 GeV).
\end{indentlist}

  {\bf    Following jet info available only if METHOD=1:}

\begin{indentlist}{ 3.50cm}{ 3.75cm}

\indentitem{CLMASJ(*)}

        REAL  : Same variables as CLMASS for jets, ordered as in TRKJET
\indentitem{CTSTAJ(*)}
        REAL  : Same variables as CTSTAR for jets, ordered as in TRKJET
\indentitem{NSECJ(*)}
        INTE   : Same variables as NSEC   for jets, ordered as in TRKJET
\indentitem{LISTTRK(*)}    
               INTE : List of all ALPHA charged tracks and V0s which
                     were considered by QBMTAG.
                     If the track is neutral, it is a V0. If a V0
                     is not in the range KFV0T-KLV0T, then it has
                     been created using YTOP, and its daughters
                     are (refitted) copies of tracks from the
                     KFCHT-KLCHT section.
\indentitem{LISTEH(*)}
        INTE : If track comtributed to 1.8 GeV hemisphere mass
                     then =1 or =2 to show which hemisphere it's in.
                     If =0, then track was rejected.
\indentitem{LISTJET(*)}
        INTE : If track comtributed to 1.8 GeV jet mass,
                     then =ALPHA track number of jet.
                     If =0, then track was rejected.
\end{indentlist}



\par
\subsection{\label{sec-QBMTAM}Remarks on QBMTAG}
\par

     If  Method=1, then one can obtain higher purities by using the
      BNEG card. This allows QIPBTAG and QBMTAG to use information from
      both +ve and -ve impact parameter tracks. (See ALNEWS 1389 in
      OFFLINE folder). If one omits this card, however, then in the same
      way as one can measure the uds efficiency of QIPBTAG using the -ve
      hemisphere tag NPROBHEMI, it is possible to create a -ve tag from
      QBMTAG for the same purpose. This can be done by calling it with
      argument NPROBTRK instead of PROBTRK.


      Please send all remarks concerning this package to :
      Ian.Tomalin@cern.ch
   

\par
\section{\label{sec-OARVSRC}QVSRCH : Secondary vertices and b-tagging}
\par
\fbox{
\parbox{4.6in}{CALL ~QVSRCH(BPOS,BSIZ,PVTX,EPVTX,DJET,BTAG,SVTX,ESVTX,
\ \  AVTX,CAVTX,JERR)}
}
\par
This routine from T. Mattison performs a primary vertex finding , a 2-jet finding
and a secondary vertices finding with an estimation of b quark tagging variables .
\par
To use it properly , one must read the complete description of the method and of the
package in the Aleph Note 92$-$173 (December 1992).
\par
 QVSRCH cannot be used when reading a Nano-Dst, since it needs the track error matrices.
\par
{\bf Input arguments:}
\begin{indentlist}{ 3.50cm}{ 3.75cm}
\indentitem{BPOS(3)}Beam position.
      BPOS(2) must be an accurate beam y coordinate,
      i.e., QVYNOM or QVTXBP(2)  (output from GET$\_$BP).
      BPOS(1),BPOS(3) are used for X and Z track cuts
       and as additional constraints on primary vertex
\indentitem{BSIZ(3)}Size of luminous region (1 sigma).
      Fixed values of .0200, .0020, 2.0 will work OK.
      Any negative element of BSIZ will suppress internal primary finding,
      in which case PVTX() and EPVTX() become input variables.
\end{indentlist}
\par
{\bf Output arguments:}
\begin{indentlist}{ 3.50cm}{ 3.75cm}
\indentitem{PVTX(3)}  Primary vertex in Aleph XYZ coordinates
\indentitem{EPVTX(3)} Error (not variance) in PVTX
\indentitem{DJET(3,2)} Two normalized XYZ direction vectors for two jets
\indentitem{BTAG(2)} B-tagging variable for the two hemispheres
      it is basically the half the chi-square difference between the
      hypothesis that all tracks in the hemisphere come from primary,
      or that some tracks come from the secondary and some from primary
      a cut at 10, or at 20 on the sum, is reasonable
\indentitem{SVTX(3,2)} Two secondary vertices in coordinate systems
      that are used internally, (one system per jet).
\begin{itemize}
\item Origins are at PVTX(), and are oriented by DJET(,)
\item The third coordinate is the decay length in the jet direction
\item The first coordinate is R-Phi-like, the second is Z-like
\end{itemize}
\indentitem{ESVTX(3,2)}Errors on SVTX() in the internal rotated system
      where geometrical correlations are small and ignored .
      May be negative, indicating a problem, not necessarily severe,
      in finding the vertex
\indentitem{AVTX(3,2)}Two secondary vertices in ALEPH XYZ coordinates
\indentitem{CAVTX(3,3,2)}Two covariance matrices for AVTX()
\indentitem{JERR}Error flag
\begin{itemize}
\item 0 All OK
\item +1 for secondary vertex 1 errors questionable
\item +2 for secondary vertex 2 errors questionable
\item +3 for questionable errors on both secondary vertices
\item +4 for no good tracks in jet 1
\item +8 for no good tracks in jet 2
\item +12 for no good tracks in either jet
\item +16 for questionable primary vertex
\item +32 for not enough tracks for primary vertex
\item +64 for no jets found
\end{itemize}
\end{indentlist}
\par
    QVSRCH performs an internal jet finding. If you don't want it, a data card:
\par
 {\bf NQVJ}
\par
      will suppress the internal jet finding,
      in which case DJET becomes an input variable which must be filled in by
      the user.
\par
\section{\label{sec-OARQPI0}QPI0DO: $\pi^0$ finding routine}
\par
The $\pi^0$ finding routine QPI0DO
is described in the report ALEPH 93$-$095 / PHYSIC 93$-$078.
It uses the photons from the ALPHA GAT section
(\ref{sec-TVAPGPC} on p.~\pageref{sec-TVAPGPC}).
 
\par
To use QPI0DO, one has simply to put in QUEVNT:
\par
\fbox{CALL QPI0DO}
\par
{\bf Input argument: none}
\par
{\bf Output arguments:}
\par
      All output arguments are returned in the ALPHA common deck GAMPI0:
\begin{verbatim}
      PARAMETER (MXPI0=200)
      COMMON / GAMPI0 / IQPI0, NTPI0, PI0MOM(4,MXPI0),ITYPI0(MXPI0),
     +                  IPI0GAM(2,MXPI0),CHIPI0(MXPI0)
\end{verbatim}
\begin{indentlist}{ 3.50cm}{ 3.75cm}
\indentitem{IQPI0}Return code
\begin{itemize}
\item  = 0  : All OK
\item  = 1  : No $\pi^0$ found
\item  $>$ 1  :    More than MXPI0 $\pi^0$ found
\end{itemize}
\indentitem{NTPI0}
                          Number of $\pi^0$  found
\indentitem{PI0MOM}
                     Refitted 4 momentum of the NTPI0  $\pi^0$
\indentitem{IPI0GAM}
                    Number of photons 1 and 2 giving the $\pi^0$ , in the GAT section
\indentitem{CHIPI0}
                    Chi2 value after refit. Set to -999. if no convergence.
\indentitem{ITYPI0}
                    $\pi^0$ type, see below:
\begin{itemize}
\item =1 : 2 photons in same PECO, with N=2 photons in the PECO
\item =2 : 2 photons in same PECO, with N$>$2 photons in the PECO
\item =3 : 2 photons in different PECO, with N=1 photons in each PECO
\item =4 : 2 photons in different PECO, with N$>$1 photons in one PECO
\end{itemize}
 
\end{indentlist}
\par
 QPI0DO uses automatically the best photons from the GAT section.
 For POTs, DSTs and MINIs $\geq 10$  the PGAC photons (sec \ref{sec-TVAPGAC}) are used.
 For MINIs $\leq 9$, these are the PGPC photons (sec \ref{sec-TVAPGPC}).
 
\par
  Beware that the mass cut used to find the $\pi^0$ candidates in QPI0DO
  photons is quite wide: this mass window is $\pm 2\sigma$  around the
  mean GAMPEX reconstructed $\pi^0$ mass and is parametrised in the internal subroutine
  PI0LIM.  For some specific analyses, a tighter cut
  might be needed, in that case one has to come back to the original
  GAMPEX photons and check their invariant mass.
\par
  Five control histograms 9001 to 9005  can optionally be filled.
  They give a display of the $\gamma-\gamma$ invariant mass versus $\gamma$ energy
   for all photon pair  types ITYPI0 (see above), in scatter plot 9000+ITYPI0 +1 ,
   plot 9001 giving the sum over all types.
\par
  To get these histograms , you need simply to put:
\par
CALL QPI0BK
\par
in your QUINIT subroutine.
 
\par
  A debug printout routine is also provided to print all $\pi^0$s found in an event;
  to call it, put:
\par
CALL PI0DEB
\par
in your QUEVNT subroutine.
 
 
\par
\section{\label{sec-OAQBEAM}QBEAMX : Size of luminous region (LEP 1 only)}
\par
\fbox{CALL QBEAMX(KRUN,KEVT,IRET,SIGX)}
\par
The ALPHA array QVTSBP (section \ref{sec-CHUNKINF}) provides an average
horizontal beam size sigma$\_$x for each year.  QBEAMX is designed
to provide run/event-dependent sigma$\_$x values for the real data
and to provide corresponding information for Monte Carlo.  The
data is divided into ``metachunks'' of typically 275 qqbar events
(about four beam position ``chunks'') for the measurements. The numerical values are
read automatically by ALPHA at initialisation time and put in internal banks VBWP,WIDE and WIDN .
 
To use QBEAMX, one must make the above subroutine call during the event analysis stage .
 
\begin{indentlist}{ 3.50cm}{ 3.75cm}
 
\indentitem{KRUN}   Input argument : run number
\indentitem{KEVT}   Input argument : event number
\indentitem{IRET}   Return code
                               (=0 if successful).  Notice that QBEAMX will
  not succeed if XGETBP (section \ref{sec-CHUNKINF}) is .FALSE..
\indentitem{SIGX}
            Output rms size (in cm) of the luminous region in the x direction.
\end{indentlist}
\par
{\bf Real data:}
The measured value of SIGX is taken from the WIDE bank.  A bias
correction (from the database bank VBWP) is applied.
\par
 
{\bf Monte Carlo: }
The value of sigma$\_$x thrown in the subroutine QFGET$\_$BP is recovered by means of
the entry QFMCSX.  NEVE, the number of events in the metachunk,
is thrown according to the distribution in the real data, as found
in the WIDN bank.  The value of sigma$\_$x is smeared according to
the resolution (as a function of NEVE, with parameters in VBWP)
and the result is returned in SIGX.  Thus the data and Monte Carlo
should be directly comparable as far as SIGX is concerned.
 
\par
\section{\label{sec-OAQSELE}QSELEP : Lepton Identification for Heavy Flavours}
\par
 
      QSELEP is a general lepton identification package
      for Heavy Flavour Physics.
      The corresponding physics results are described in the paper:
 
      `` Heavy quark tagging with leptons in the ALEPH detector ''
 
      Nucl. Instr. Meth. {\bf A 346} ( 1994 ) pp 461 $-$ 475.
 
 
      QSELEP is built basically from the LEPTAG package    (note ALEPH 92-101 / PHYSICS 92-090)
      written by Mark Parsons ,
      adapted to ALPHA by Ingrid Ten Have and M-N Minard in November 1994. However it contains also
      most features of HEVLEP, the physics results and properties of which are described in detail in the note
      ALEPH 94-055 / PHYSIC 94-049. Due to technical reasons, the CALPOIDS package, auxiliary to HEVLEP,
      is not included into QSELEP and is still on UPHY.
 
      The Monte-Carlo part of QSELEP: QTRUTH,  is described in the next section (\ref{sec-OAQTRUT}).
 
\subsection{\label{sec-QSELCA}How to call QSELEP}
\par
         You don't have to call explicitely the QSELEP routine: either you
         get the results directly from MINI banks, or you trigger its execution
         through data cards as explained in \ref{sec-QSELCU}.
 
{\bf When reading a MINI version $\geq 10$:}
 
                  If you are happy with the standard cuts used to build the
         MINIs, you have nothing to do: the QSELEP results are available
         directly through the ALPHA variables described afterwards in sec
                             \ref{sec-QSELTL} to \ref{sec-QSELJT}.
 
                  If you want to use tighter cuts, please refer to sec \ref{sec-QSELCU}.
         This may be useful e.g. to use the CALPOIDS weighting (in UPHY).
 
{\bf When reading a POT or a DST or an old MINI:}
 
         If you want to call QSELEP yourself on a POT or DST (or on old MINIs
         made with MINILIB version 9.0 and before): you have to put in your
         CARDS file the  data cards described in sec \ref{sec-QSELCU}. You don't have to call
         QSELEP in your Fortran program: the call and all the necessary
         initialisations will be triggered by the QLID data card.
 
         Then you may access the tagged leptons informations by the new ALPHA
         statement functions described below in sec
                             \ref{sec-QSELTL} to \ref{sec-QSELJT}.
 
\par
\subsection{\label{sec-OAQSELW}Important warning if you call QSELEP yourself:}
 
\par
         If you call QSELEP when reading a POT/DST, or when reading a MINI
         with ``non-standard'' cuts (see sec \ref{sec-QSELCU})  be aware that:
 
         QSELEP RESETS ALL ALPHA TRACK LOCKS and uses them internally.
         It uses also internally ALL ALPHA TRACK USER WORDS. Therefore you
         must be very careful if your analysis needs these locks and/or user
         words, and if you call QSELEP through the QLID card and user's cuts.
 
 
\subsection{\label{sec-QSELOU}Output of QSELEP}
 
      The output of QSELEP consists of two banks PDLT,PLJT for real data, plus an additional
      one PMLT for MCarlo data. These banks are made using the default  cuts
      described in \ref{sec-QSELCU} and are written on MINIs versions $\geq 10$.
      They are also available on NANOs versions $\geq115$.
 
      User access to these banks is done through the ALPHA variables and statement functions
      described below in \ref{sec-QSELTL} to \ref{sec-QSELJT}.
 
      When a MINI version $\geq 10$ is written,
      all the cuts used by QSELEP are written onto a run header
      bank, PLSC. This bank is used to set the default cuts for QSELEP when reading this MINI with ALPHA.
 
\subsection{\label{sec-QSELCU}Default cuts and users cuts}
 
         If you are not happy with the standard cuts used when running QSELEP to make a MINI
         (version $\geq 10$) , or if you want to create
                          the QSELEP information from a POT/DST/old MINI:
            you have first to put in your CARDS file the two data cards
 
  {\bf   QLID}
 
  {\bf   EFLJ}
 
         If you don't provide any of the other data cards below, QSELEP will
         not be called when reading MINIs $\geq 10$ since the cuts are the same as used to build it.
         If you run on a POT/DST/old MINI, QSELEP will be called using
         the default values described below.
 
         If you want to use other cuts, you have to put in your CARDS file
         any combination of the data cards
         described below.  
         The data cards described below correspond to the default values used
         to produce MINIs.  Comments are self-explanatory.
\begin{verbatim}
*
*--  User card for general charged track selection
*
*-- * #tracks e_frac #tpc   d0    z0     cos(theta_max)
LSHD    5      0.10    4    2.0   10.0      0.95
*
*--   User card to select electron candidates in charged tracks
*--
*-- *  #tpc   d0    z0    cos(the_max)   pcut  p_tcut
LSET    5    2.0   10.0      0.95        2.0    0.0
*--
*--   User card electron identification cuts
*--
*-- *  rt_min rt_max rl_min rl_max #wires ri_min ri_max
LSEC   -3.0   999.0  -3.0   3.0      0    -999.  999.
*--
*--   User card rejection of photon conversions and Dalitz decays
*--
*-- * XY_max Z_max  M_max   W_conv(1=write,0=reject)
LSCV   1.0    1.0   0.020   1
*--
*--   User card to select muon candidates in charged tracks
*--
*-- * #TPC  d0    z0    cos(the_max)  pcut   p_tcut
LSMT    5   2.0  10.0      0.95       3.0     0.0
*--
*--   User card muon identification cuts
*--
*-- *  QMUIDO return flags accepted
LSMC   13  14
*--
*--   User card jet cuts
*--
*-- * #objects/jet  Max P(lepton)/E(jet) #jets   Ycut      Jrej
*--                                                       1=keep,0=cut
LSJT      5                 0.9            2     0.004       1
 
\end{verbatim}
 
\subsection{\label{sec-QSELRS}Access to QSELEP results }
\par
         In all circumstances :
 
           - either if you have called QSELEP yourself through QLID + cards
 
           - or if you read a MINI / NANO containing the QSELEP output banks
 
         you can access the results of QSELEP through the ALPHA variables
         described in the following subsections .
 
\subsection{\label{sec-QSELTL}Properties of selected tagged Leptons }
\par
              When doing a loop on all reconstructed charged tracks :
 
                    DO ITK=KFCHT,KLCHT
 
              you can access the QSELEP results as follows :
 
      ALPHA Track ITK is a Lepton tagged by QSELEP if :
 
           {\bf XLEPTG(ITK) = .TRUE.}
 
          If yes , then :
 
\begin{indentlist}{ 3.50cm}{ 3.75cm}
\indentitem{KLEPPA(ITK)}    QSELEP lepton  type :
\par
\begin{itemize}
 
\item                 2 = e+
\item                12 = e+ in crack region
\item                22 = e+ in overlap region
\item                 3 = e-
\item                13 = e- in crack region
\item                23 = e- in overlap region
\item                 5 = mu+ QMUIDO IDF=14
\item                15 = mu+ QMUIDO IDF=13
\item                25 = mu+ QMUIDO IDF=11
\item                35 = mu+ QMUIDO IDF=15
\item                45 = mu+ all other QMUIDO flags
\item                 6 = mu- QMUIDO IDF=14
\item                16 = mu- QMUIDO IDF=13
\item                26 = mu- QMUIDO IDF=11
\item                36 = mu- QMUIDO IDF=15
\item                46 = mu- all other QMUIDO flags
\end{itemize}
\indentitem{KLEPJT(ITK)}    ALPHA "track" Number of nearest jet
                          in LJET section of ALPHA
                          The properties of this jet can be accessed directly
                          by standard ALPHA functions
 
                          e.g. PXJET = QX(KLEPJT(ITK))
\indentitem{QLEPPI(ITK)}  Transverse momentum respect to the jet , lepton included
\indentitem{QLEPPE(ITK)}  Transverse momentum respect to the jet lepton excluded
\indentitem{KLEPVP(ITK)}  Flag giving the validity of the Pt calculation :
\par
\begin{itemize}
 
\item                0 = OK
\item               10 = No E-flow track found
\item               20 = No valid jet
\item               30 = Not enough objects in jet
\end{itemize}
\end{indentlist}
 
\subsection{\label{sec-QSELTH}QTRUTH flags for recontructed leptons (Mcarlo only)}
\par
 
              When reading MCarlo events, results from the QTRUTH part of QSELEP
              can be accessed as follows :
 
              When doing a loop on all reconstructed charged tracks :
 
                    DO ITK=KFCHT,KLCHT
 
          ALPHA Track ITK is a MCARLO tagged lepton with QTRUTH results if :
 
            {\bf XLEPTH(ITK) = .TRUE.}
 
          If yes , then :
 
\begin{indentlist}{ 3.50cm}{ 3.75cm}
\indentitem{KLEPFL(ITK)}      Primary quark Flavour of the event
 
                              1=d, 2=u, 3=s, 4=c, 5=b
\indentitem{KLEPPO(ITK)}      Flavour of the quark from gluon splitting
 
                              4 or 5 if the chain is originating from a
                                     c or a b quark respectively, which
                                     are coming from gluon splitting
 
                              0      otherwise
\indentitem{KLEPCH(ITK)}     H.F. process originating the decay chain
 
\begin{itemize}
 
\item                         -2    b $\rightarrow$ cbar u ; cbar $\rightarrow$ X
\item                         -1    b $\rightarrow$ u X
\item                          1    b $\rightarrow$ c X
\item                          2    b $\rightarrow$ cbar c ; cbar $\rightarrow$ X
\item                          3    b $\rightarrow$ c $\rightarrow$ X
\item                          4    c $\rightarrow$ X
\item                          5    fragmentation or uds
\end{itemize}
\indentitem{KLEPSP(ITK)}     End of the decay chain
 
\begin{itemize}
\item                          0   lepton from H.F. hadron
\item                          1   K $\rightarrow$ l
\item                          2   pi $\rightarrow$ l
\item                          3   gamma $\rightarrow$ l
\item                          4   J/psi $\rightarrow$ l
\item                          5   psi' $\rightarrow$ l
\item                          6   pi0 $\rightarrow$ l
\item                          7   other $\rightarrow$ l
\item                          8   K
\item                          9   pi
\item                         10   other
\end{itemize}
\indentitem{KLEPLE(ITK)}     Track identity
 
\begin{itemize}
\item                         -1   unidentified
\item                          0   non lepton
\item                          1   mu
\item                          2   tau $\rightarrow$ mu
\item                          3   electron
\item                          4   tau $\rightarrow$ e
\end{itemize}
\indentitem{KLEPME(ITK)}     Identity of the decaying b particle (if any)
 
\begin{itemize}
\item                         -2   "oscillated" Bs
\item                         -1   "oscillated" Bd
\item                          0   b baryon or non b particle
\item                          1   Bd
\item                          2   Bs
\item                          3   B+
\end{itemize}
\indentitem{KLEPKT(ITK)}     FKIN   track number of the lepton
\end{indentlist}
 
\subsection{\label{sec-QSELJT}Jets built by QSELEP : ``LJET'' section}
\par
 
                  As QSELEP builds its own jets , a new ALPHA section is built
         to access them through standard ALPHA calls . These jets are stored
         as standard ALPHA ``tracks" in a new ALPHA ``track" section .
 
         Name of the Section : 'LJET'
\begin{itemize}
\item        KLFJET     = ALPHA ``track'' number of 1st  QSELEP jet
\item        KLLJET     = ALPHA ``track'' number of last QSELEP jet
\item        KLNJET     = number of QSELEP jets
\item        KLJTNO(I)  = number of objects inside QSELEP jet I
\end{itemize}
         The access to the jet properties can be done either by a DO LOOP :
\begin{verbatim}
             DO IJET = KLFJET,KLLJET
                PXJET = QX(IJET)
                PYJET = QY(IJET)
                PZJET = QZ(IJET)
                 EJET = QE(IJET)
                 NOBJ = KLJTNO(IJET)
             etc..
 
     or by an implicit loop on "particles" with name 'LJET' of class KRECO :
 
             IJET = KPDIR('LJET',KRECO)
        10   IF (IJET.EQ.0) GO TO 100
                PXJET = QX(IJET)
                .....
             IJET = KFOLLO('LJET')
             GO TO 10
        100  CONTINUE
\end{verbatim}
 
 
\subsection{\label{sec-QSELED}New EDIR classes from QSELEP }
\par
 
         When an output file is written in an ALPHA job which called QSELEP ,
         two new EDIR classes are defined on the output class word in bank REVH
 
              Class 27 :  Electron selected ( Rt $\geq -3.$ , $ -3. \leq$  Rl $ \leq 3.$ )
 
              Class 28 :  Muon selected ( QMUIDO flag = 13 or 14 )
 
     Events with these EDIR classes may be accessed directly through the standard  CLAS data card ( \ref{sec-DCEVD} on p.
     ~\pageref{sec-DCEVD} )  when reading a MINI version $\geq 10$ .
 
\par
\section{\label{sec-OAQTRUT}QTRUTH : History of a recontructed MCarlo track:}
\par
\fbox{CALL QTRUTH(ITK,IFLA,IPOP,ICHAIN,ISPLIT,LEPID,IBMES)}
\par
 
 
     QTRUTH is part of the QSELEP package described in the previous section. However it may be called
     independently, outside the context of QSELEP.
 
     QTRUTH gives information about the ``history" of
     a RECONSTRUCTED MCarlo track ITK. It works on Monte Carlo
     hadronic events generated with JETSET (or coming from an
     event generator interfaced with JETSET). The information
     provided is aimed to the classification of candidate
     leptons for heavy flavour tagging, but may be useful
     also in many other cases.
 
 
\begin{indentlist}{ 3.50cm}{ 3.75cm}
\indentitem{ITK}
      Input argument :   Alpha index of the reconstructed track
\indentitem{IFLA}      Primary quark Flavour of the event
 
                              1=d, 2=u, 3=s, 4=c, 5=b
\indentitem{IPOP}      Flavour of the quark from gluon splitting
 
                              4 or 5 if the chain is originating from a
                                     c or a b quark respectively, which
                                     are coming from gluon splitting
 
                              0      otherwise
\indentitem{ICHAIN}   H.F. process originating the decay chain
 
\begin{itemize}
\item                         -2    b $\rightarrow$ cbar u ; cbar $\rightarrow$ X
\item                         -1    b $\rightarrow$ u X
\item                          1    b $\rightarrow$ c X
\item                          2    b $\rightarrow$ cbar c ; cbar $\rightarrow$ X
\item                          3    b $\rightarrow$ c $\rightarrow$ X
\item                          4    c $\rightarrow$ X
\item                          5    fragmentation or uds
\end{itemize}
\indentitem{ISPLIT}    End of the decay chain
 
\begin{itemize}
\item                          0   lepton from H.F. hadron
\item                          1   K $\rightarrow$ l
\item                          2   pi $\rightarrow$ l
\item                          3   gamma $\rightarrow$ l
\item                          4   J/psi $\rightarrow$ l
\item                          5   psi' $\rightarrow$ l
\item                          6   pi0 $\rightarrow$ l
\item                          7   other $\rightarrow$ l
\item                          8   K
\item                          9   pi
\item                         10   other
\end{itemize}
\indentitem{LEPID}    Track identity
 
\begin{itemize}
\item                         -1   unidentified
\item                          0   non lepton
\item                          1   mu
\item                          2   tau $\rightarrow$ mu
\item                          3   electron
\item                          4   tau $\rightarrow$ e
\end{itemize}
\indentitem{IBMES}   Identity of the decaying b particle (if any)
 
\begin{itemize}
\item                         -2   "oscillated" Bs
\item                         -1   "oscillated" Bd
\item                          0   b baryon or non b particle
\item                          1   Bd
\item                          2   Bs
\item                          3   B+
\end{itemize}
\end{indentlist}
 
\par
\section{\label{sec-OAMCMAT}MCMATCH : Matching of a Monte Carlo truth track with a reconstructed track}
\par
\fbox{CALL MCMATCH(IMCTRU,IRECTR,IFLAG)}
\par
 
 
     The routine MCMATCH checks if a ``truth" track at the FKIN (generator) level is reconstructed or not
     after GALEPH and JULIA. See the note of D.Brown ALEPH 94-135 (PHYSIC 94-118) for more details.
 
 
\begin{indentlist}{ 3.50cm}{ 3.75cm}
\indentitem{IMCTRU}
      Input argument :   Alpha index of a ``truth"  track (between KFMCT and KLMCT)
\indentitem{IRECTR}      Alpha index of the matching reconstructed charged track (between KFCHT and KLCHT)
 
                         = 0 if no matching found
\indentitem{IFLAG}   Flag describing the matching status:
 
\begin{itemize}
\item                         -3    No match (lost ``truth" track)
\item                         -2    Track decays before making hits
\item                         -1    No well matching track
\item                          0    MC track double counted
\item                          1    Normal good match from primary vertex
\item                          2    Normal good match from secondary vertex
\item                          2    Normal good match from V0 decay
\item                          4    Kink fit as 2 tracks
\item                          5    Kink fit as 1 track
\item                          6    Normal good match from material interaction
\end{itemize}
\end{indentlist}
 
\par
\section{\label{sec-OAJULMAT}JULMATCH : Matching of a reconstructed charged track with a MC ``truth" track}
\par
\fbox{CALL JULMATCH(IALTRK,IMCTRU,IFLAG)}
\par
 
 
     The routine JULMATCH checks if a reconstructed track comes from a
     ``truth" track at the FKIN (generator) level.
                             See the note of D.Brown ALEPH 94-135 (PHYSIC 94-118) for more details.
 
 
\begin{indentlist}{ 3.50cm}{ 3.75cm}
\indentitem{IALTRK}
      Input argument :   Alpha index of a reconstructed charged track (between KFCHT and KLCHT)
\indentitem{IMCTRU}      Alpha index of the ``truth"  track (between KFMCT and KLMCT)
 
                         = 0 if no matching found (IFLAG described below is -2 or -3)
\indentitem{IFLAG}   Flag describing the matching status:
 
\begin{itemize}
\item                         -3    Hits not related to MC true hits
\item                         -2    Hits mixed from several MC true particles (not decays)
\item                         -1    Spiral branch or calorimetric reflection
\item                          0    Double counted particles (2 tracks for 1 true particle)
\item                          1    True track comes from true primary
\item                          2    True track comes from HF decay
\item                          3    True track comes from neutral decay
\item                          4    True track comes from charged decay (split kink)
\item                          5    Kink fit as 1 track
\item                          6    True track comes from material interaction
\item                          7    Unknown
\end{itemize}
\end{indentlist}
 
\par
\section{\label{sec-OAVDHMAT}VDHMATCH : Counting of matching VDET hits for a given track}
\par
\fbox{CALL VDHMATCH(IALTRK,IMCTRU,NTRUE,NWRONG,NMISS,NDOUBLE)}
\par
 
 
     The routine VDHMATCH must be called only AFTER a call to JULMATCH.
                             See the note of D.Brown ALEPH 94-135 (PHYSIC 94-118) for more details.
 
     This routine cannot be called when running on a MINI, since it needs the detailed VDET hits
which are not stored on the MINIs.
 
\begin{indentlist}{ 3.50cm}{ 3.75cm}
\indentitem{IALTRK}
      Input argument :   Alpha index of a reconstructed charged track (between KFCHT and KLCHT)
\indentitem{IMCTRU}
      Input argument :   Alpha index of the matching ``truth"  track as given by JULMATCH
                         (must not be 0)
\indentitem{NTRUE}   Number of true VDET hits on this track
\indentitem{NWRONG}   Number of true hits on the reconstructed track which truly came from a track other than IMCTRK
                      (split z/r$\phi$)
\indentitem{NMISS}   Number of true VDET hits not used in reconstruction (split z/r$\phi$)
\indentitem{NDOUBLE}  Number of true hits on the reconstructed track which
                      are associated with 2 or more true Monte Carlo tracks (split z/r$\phi$)
\end{indentlist}
 
\par
\section{\label{sec-ELEP2}QWHICH$\_$EN: to know how the LEP energy QELEP was found}
\par
\fbox{CALL QWHICH$\_$EN(ISTAT)}
\par
 
 
     The output argument ISTAT may be used to know how the  center-of-mass LEP energy QELEP was found.
     For most LEP2 events, this energy is defined in 15 minutes time slices by the
     LEP Energy Working group. It is available only for high energy runs of 1996 and after.
                             See the note of P.Bright-Thomas ALEPH 97-029 (PHYSIC 97-024) for more details.
 
 
\begin{indentlist}{ 3.50cm}{ 3.75cm}
\indentitem{ISTAT}
     Output argument:   flag telling how QELEP was obtained.
 
\begin{itemize}
\item                        = 0    No QELEP energy found! This should NEVER happen !

\item                        = 1    QELEP  is obtained from the 'RLEP' bank. This is always the case for Monte Carlo events.       
                                    For real data, this should be very rare, and only if all other methods described below 
                                    have failed.

\item                        = 2    QELEP  is an averaged value over the current LEP fill.
                                     This is the case for 1989 to 1992  real data (from database bank LFIL),
                                     or for events in some very short runs taken in 1993 
                                     and after (from database bank RNF2).

\item                        = 3    QELEP  is an average value over the current run, taken from the database bank RNR2. 
                                    This is the case for
                                    LEP 2 data if no time slice value was available for the event (mostly events taken before or
                                    after the official start/stop time of a LEP fill, as recorded in the LEP Energy working group
                                    tables).

                                     For LEP 1 data from 1993 to 1995 included, the values are run averages computed from the LEP
                                     Working Group values.

\item                        = 4    QELEP was obtained from the 15 minutes time-slice energy determinations for ALEPH, from the LEP
                                    energy working group, stored in the database bank RNL2. These time-slices exist
                                    only for LEP2 High Energy data for 1996 and after. 

\end{itemize}
\end{indentlist}


\par
\section{\label{sec-EBSPOT}QWHICH$\_$BP: to know how the beam spot was found}
\par
\fbox{CALL QWHICH$\_$BP(ISTAT)}
\par


     The output argument ISTAT may be used to know how the beam spot was found (see
 \ref{sec-CHUNKINF} on p.~\pageref{sec-CHUNKINF}).

\begin{indentlist}{ 3.50cm}{ 3.75cm}
\indentitem{ISTAT}
     Output argument :   flag telling how the beam spot parameters were obtained.

\begin{itemize}
\item                        = 0    No beam position found! This should NEVER happen !

\item                        = 1    obtained from the run header bank  JSUM.
                                    This should be very rare, and only if all other methods described below
                                    have failed.

\item                        = 2    obtained from an averaged value over the current LEP fill (database bank LFIL).
                                    This is the case only if the two methods described below have failed.


\item                        = 3    obtained from an averaged value over the current run  (database bank RXYZ).
                                    This happens  only for LEP1 data in old processings (JULIA version below 300) 
                                    if no event-chunk position exists. 

\item                        = 4     obtained from the event-chunk beam positions (run header bank ALPB).

\end{itemize}
\end{indentlist}


\par
\section{\label{sec-BOMSTAT}QFILBP$\_$STATUS: to know how the LEP beam position with BOMS was found:}
\par
\fbox{CALL QFILBP$\_$STATUS(ISTAT)}
\par


     The output argument ISTAT may be used to know how the LEP beam position with BOMS was found.
     This implies that the data card BOBS was used (see \ref{sec-DCBOBS} on p.~\pageref{sec-DCBOBS}).

\begin{indentlist}{ 3.50cm}{ 3.75cm}
\indentitem{ISTAT}
     Output argument :   flag telling how the LEP beam position with BOMS was found.

\begin{itemize}
\item                        = 0    No standard chunk-by-chunk beam position (XGETBP is    FALSE.)\\
                                     All positive values below are obtained for XGETBP = .TRUE.

\item                        = 1    VDET (x)       VDET (y)
\item                        = 2    BOM+QSO (x)   VDET (y)
\item                        = 3    VDET (x)      BOM+QSO (y)
\item                        = 4    BOM+QSO (x)   BOM+QSO (y)
\item                        = 5    Average (x)    VDET (y)
\item                        = 6    VDET (x)       Average  (y)
\item                        = 7    Average (x)   Average  (y)


\end{itemize}
\end{indentlist}




\par
\section{\label{sec-EWSU2}QEWSUM : ECAL Wire Energy on even/odd wire planes}
\par
\fbox{CALL QEWSUM(EWEVEN,EWODD,IER)}
\par


     The routine QEWSUM 
gives the energy on ECAL wires summed over even/odd
      planes. These quantities are available from POTs/DSTs but cannot be obtained from MINIs 
version 102 and before; on MINIs they are available only for datasets written with MINI version 
       103 and after, since May 1997.
                        

\begin{indentlist}{ 3.50cm}{ 3.75cm}
\indentitem{EWEVEN(36)}
     Output array:
      sum of wire energies on even wire planes for the 36 ECAL modules
\indentitem{EWODD(36)}
        Output array: 
      sum of wire energies on odd  wire planes for the 36 ECAL modules 

\indentitem{IER}
     Output argument :   

= 1 if no banks of wire energies for this event

             = 0 if all OK
             
\end{indentlist}

      In the arrays EWEVEN and EWODD above, ECAL modules 1 $-$ 12 refer to endcap A, 13 $-$ 24 to the barrel, and 25
$-$ 36 to endcap B.

\par
\section{\label{sec-QPCOR}QPCORR : Correction of charged particle momenta (``Sagitta correction")}

     This function has been provided by I. Tomalin and introduced in ALPHA 122.44 in March 1998.
 This is the so called ``sagitta correction".
 It corrects particle momenta for the effects of residual distortions in
     the central tracking detector.
     It depends upon $\cos(\theta)$ and the year of data taking.
  
    Every year, even after the detector alignment is finished and
     corrections have been made for field distortions etc., it is found
     that Ebeam/P in $Z0 \rightarrow  \mu^+ \mu^-$ events is not precisely 1, presumably
     because of residual distortions.
     (Typically, in the region $|\cos(\theta)| > 0.9$, Ebeam/p is about 0.94
     for +ve tracks and 1.06 for -ve tracks. Elsewhere, Ebeam/p is usually
     consistent with 1 to within a percent or so. The effect is not quite
     forward-backward symmetric).

     The relative bias in momentum is proportional to P, so most people
     analysing hadronic events can ignore it. Exceptions include analyses
     using the ECAL electron identifiers in the region $|\cos(\theta)| > 0.9$
     or analyses which are very sensitive to systematic biases in the
     momenta (e.g. jet charge, $\tau$ polarization).
   
   QPCORR  provides a correction for the momenta based upon
     Ebeam/P measurements in $Z0 \rightarrow  \mu^+ \mu^-$ events. It assumes that the
     corrections for -ve and +ve particles are equal in size, but of
     opposite sign. This is observed to be true, apart from a constant
     offset, Ebeam/p = 1.002, which is also present in the MC and so not
     corrected for.

     In principle, the correction depends upon your track selection
     cuts, but providing that the corrections have a small effect on your
     analysis, you can ignore this. Arguement NVDET does correct for this
     to first approximation however.
     
\par
\fbox{FUNCTION  QPCORR(ITRK,ASCALE,NVDET,ERCORR)}

{\bf Input arguments:}

\begin{indentlist} { 3.50cm}{ 3.75cm}

\indentitem{ITRK}     Integer: ALPHA track number.

\indentitem{ASCALE}   Logical: Set .TRUE. if ALPHA variables like QX,QY,QZ
                          are to be rescaled by calling QVSCAL.\\
                          N.B. Don't call this routine twice for the
                          same track with ASCALE = .TRUE. !
\indentitem{NVDET}   Integer: If your analysis only uses tracks with at
                          least 1 VDET hit, set NVDET=1; otherwise =0.
\end{indentlist}
{\bf Output arguments:}
\begin{indentlist} { 3.50cm}{ 3.75cm}

\indentitem{QPCORR}   Real    : Scale factor applied/to be applied to momentum.
\indentitem{ERCORR}   Real    : Statistical error on this factor.
    
\end{indentlist}

{\bf Important remark: data card PCOR allows to call QPCORR automatically:}

\begin{indentlist}{ 2.50cm}{ 2.75cm}
\indentitem{PCOR 0}Call QPCORR  for all charged tracks 
\indentitem{PCOR 1}Call QPCORR  only for charged tracks having VDET coordinates 
\end{indentlist}





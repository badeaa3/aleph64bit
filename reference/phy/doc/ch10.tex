\chapter{\label{sec-QJ}Event Topology Routines}
\par
All of the subroutines described in this chapter perform loops over
tracks or
particles.
The arguments and loop algorithms are similar
for all of these subroutines, and are described in detail in
Section \ref{sec-QJA}.
The ``tracks'' to be considered are
selected with the routines QJOPTR (for reconstructed tracks) and QJOPTM
(for Monte Carlo tracks); these routines also specify tracks to be
used by the Lorentz transformation routine QTCLAS
(see \ref{sec-QTC}).
In addition, the LOCK routines described in Section~\ref{sec-QL}.
can be used to exclude tracks from analysis by the QJxxxx routines
described in this chapter.
 
\section{\label{sec-QJOP}Options for ``QJxxxx'' routines}
\par
 
\subsection{\label{sec-QJORE}Set option for reconstructed objects}
\par
\fbox{CALL QJOPTR ('reco$-$option', 'additional')}
\par
\begin{indentlist}{ 2.50cm}{ 2.75cm}
\indentitem{Input arguments:}
\indentitem{'reco$-$option'}One of the following options:
\begin{itemize}
\item {\bf 'RE'}: ``REconstructed'' tracks
({\bf default}; see \ref{sec-AL})
\item {\bf 'CO'}: Calorimeter Objects
\item {\bf 'CH'}: CHarged tracks
\item {\bf 'EF'}: ENFLW or mask energy flow objects depending on ELFW option;
see Ch.~\ref{sec-EF}.
\item {\bf 'EJ'}: YCUT=0.003 jets based on objects in EF section;
see~\ref{sec-EFLWM}.
\item {\bf 'PC'}: PCPA-based energy flow using PCPA neutral objects and
selected charged tracks; see~\ref{sec-EFLWP}.
\item {\bf 'AL'}: All objects (charged tracks, cal. objects,
ECAL objects, HCAL objects, V0s, V0 daughter tracks, etc.).
If not applied skillfully together
with LOCK, many objects will be counted twice.
\item {\bf 'NO'}: NO object. Only objects specified by
'additional' (see below) will be taken into account.
\end{itemize}
\indentitem{'additional'}Particle name of one or several additional
particle(s) to be analyzed. If no additional
particles are
to be considered, the argument ' ' must be given
(\eg, CALL QJOPTR('CO',' ')).
\end{indentlist}
The following example would cause the QJ routines to consider charged
tracks and all particles called MISS$-$VECTOR; MISS$-$VECTOR might
be a
pseudo$-$particle created by one of the routines described later in
this
chapter.
\begin{verbatim}
      CALL QJOPTR ('CH', 'MISS-VECTOR').
\end{verbatim}
Specifying additional reconstructed particles (QJOPTR) has no impact
on MC particles (QJOPTM) and vice versa.
 
\subsection{\label{sec-QJOMC}Set option for MC particles}
\par
\fbox{CALL QJOPTM ('MC$-$option', 'additional')}
\par
\par
\begin{indentlist}{ 2.50cm}{ 2.75cm}
\indentitem{'MC$-$option'}One of the following options:
\begin{itemize}
\item {\bf 'VI'}: Only particles with
a stability codes $>$ 0. VI stands for `best chance to be visible'.
({\bf default}: see \ref{sec-TVASC})
\item {\bf 'EP'}: Only particles with stability codes 1, 2, or $-$3.
EP stands for energy-momentum conservation.
\item {\bf 'AL'}: All objects. If not applied carefully together
with LOCK, many objects will be counted twice.
\item {\bf 'NO'}: No object. Only objects specified by
`additional' will be taken into account.
\end{itemize}
\indentitem{'additional'}Same as for QJOPTR.
\end{indentlist}
 
\section{\label{sec-QL}Lock tracks / subsamples of tracks}
\par
The ``LOCK'' routines described here make it possible to exclude tracks
from analysis by the routines (QJxxxx) described in this chapter.
This
feature can be used both to flag background tracks and to
restrict the analysis to a subsample of all tracks (\eg, to consider
only tracks which contribute to a given jet).
In any user routine, you may test the lock status of a given
track ITK with {\bf XLOCK(ITK)} which is {\bf .TRUE.}
if the track has been locked.
\par
Every track has three independent locks: one simple one (QLTRK)
and two more complicated ones (QLOCK and QLOCK2)
with a broader scope of
applications. If desired, several locks can be used simultaneously.
A
track is considered ``unlocked'' if and only if all three locks are
open.
\par
Opening and closing locks is done only in user routines; no track
is locked unless it is explicitly locked by the user.
 
\subsection{\label{sec-QLI}Lock a single ``track''}
\par
\fbox{CALL QLTRK (ITK)}
\par
\par
\begin{indentlist}{ 3.25cm}{ 3.50cm}
\indentitem{ITK}ALPHA ``track'' number
\indentitem{Remarks:}
In contrast to the other locks described below, QLTRK locks
the object ITK and its direct copies only (including the same
object with a different vertex assignment)
$-$$-$ no other associated
objects are affected.
\end{indentlist}
\subsection{\label{sec-QLU}Unlock a single ``track''}
\par
\fbox{CALL QLUTRK (ITK)}
\par
\par
\begin{indentlist}{ 3.25cm}{ 3.50cm}
\indentitem{ITK}ALPHA ``track'' number
\indentitem{Remark:}
QLUTRK opens only the lock set by QLTRK. If another lock is
still closed, the track remains locked.
\end{indentlist}
\subsection{\label{sec-QLO}Lock a track ``family''}
\par
\fbox{CALL QLOCK (ITK)}
\par
\par
\begin{indentlist}{ 3.25cm}{ 3.50cm}
\indentitem{ITK}ALPHA ``track'' number
\end{indentlist}
The family of track ITK consists of:
\begin{itemize}
\item The track ITK itself.
\item All copies of track ITK which have been made or will be made,
including Lorentz boosts of ITK.
\item For charged tracks, all associated cal. objects; for cal. objects,
all associated charged tracks.
\item For reconstructed tracks, all tracks based on the same reconstructed
object but assigned to different vertices, used with different mass
hypotheses, etc..
\item Daughters, granddaughters, great$-$granddaughters, ... ; i.e.,
all
kinship in
descending line.
\item Mothers, grandmothers, great$-$grandmothers, ... ; i.e., all
kinship in
directly (!) ascending line. If you use QLOCK for declaring a
reconstructed particle to be background, all its ancestors (composite
particles based on it) are implicitly declared to be background.
\item Jets and other ``pseudo particles'' described in
\ref{sec-QJA}. If you lock a
jet, you lock all contributing particles. If you lock a particle,
you
lock all jet vectors to which the particle belongs.
To lock all particles not belonging to a jet, user QLREV described
below.
\end{itemize}
Reconstructed tracks and MC truth are treated separately;
locking a reconstructed track has no effect on any MC track and vice
versa. Lock does not work if you mix up reconstructed tracks and MC.
 
\subsection{\label{sec-QLZ}Unlock tracks (locked with QLOCK)}
\par
\fbox{CALL QLZER (IREMC)}
\par
\begin{indentlist}{ 3.25cm}{ 3.50cm}
\indentitem{IREMC }= KRECO for reconstructed tracks and
KMONTE for MC truth
\end{indentlist}
Note that
the lock algorithm works for all Lorentz frames simultaneously, and
that
the specification of a particular frame is NOT allowed (in contrast
to \ref{sec-ADI}). Reconstructed objects and Monte Carlo objects
are treated separately.
QLZER opens the lock QLOCK for all tracks. Tracks may remain locked
if
other locks are still closed.
It is not possible to remove the lock QLOCK for a single track.
Using two locks
simultaneously (see \ref{sec-QL2}) should provide
all the facilities that are needed.
 
\subsection{\label{sec-QLR}Reverse the lock state (corresponding
to QLOCK)}
\par
\fbox{CALL QLREV (IREMC)}
\par
\par
\begin{indentlist}{ 3.25cm}{ 3.50cm}
\indentitem{IREMC }(see \ref{sec-QLZ}).
\end{indentlist}
\begin{itemize}
\item All unlocked tracks will be locked.
\item All locked tracks will be unlocked provided that there is no
other
closed lock and, for composite particles, that there is no locked
daughter, granddaughter, ... after the QLREV operation.
\end{itemize}
\par
Calling QLREV a second time reestablishes the initial lock state.
The
mnemonic symbol XLREV(IREMC) is set to .TRUE. if the lock state is
reversed. At the begin of the event
processing and after calling QLZER(IREMC), XLREV(IREMC) is .FALSE..
 
\subsection{\label{sec-QL2}Second Lock}
\par
\fbox{CALL QLOCK2(ITK)}
\par
\par
QLOCK2 works in the same way as QLOCK.
If one of these locks is used to flag background tracks, the other
one
can be used to select subsamples of the non$-$background tracks.
Also available:
CALL QLZER2 (IREMC),
CALL QLREV2 (IREMC), and the logical function XLREV2(IREMC).
 
\section{\label{sec-QJA}Add momenta of all particles of a given class}
\par
\fbox{CALL QJADDP (SCALAR, `vector$-$name', ICLASS)}
\par
\par
For adding momenta of a few particles, see \ref{sec-QVA}.
(NOTE: All of the QJxxxx routines have similar arguments. The
arguments are explained fully in this explanation of QJADDP.)
\par
\subsection{\label{sec-QJI}Input argument}
\par
\begin{indentlist}{ 2.50cm}{ 2.75cm}
\indentitem{ICLASS}
Class = KRECO or KMONTE or a Lorentz frame identifier (see
\ref{sec-ADI}).
If ICLASS is KRECO, note that initially all charged particles
have the pion mass and all neutral objects have mass = 0. This can
be modified by CALL QVSETM (see \ref{sec-QVM}).
If ICLASS refers to a Lorentz frame,
particles not boosted into the frame are ignored without
notification. The routine QTCLAS (see \ref{sec-QTC}) performs a Lorentz
transformation of all tracks belonging to a class. If a particle
has been boosted several times into the same frame, the most
recently boosted hypothesis will be used (see remarks in
\ref{sec-ADS}).
\end{indentlist}
 
\subsection{\label{sec-QJR}Results}
\par
A scalar result is stored in the first subroutine argument. In
QJADDP, the scalar result is the 3$-$momentum sum of all particles.
An output vector is specified by its name, which is
the second subroutine argument `vector$-$name'. If you are
interested in the scalar result only and not in the output vector,
specify a blank space ` '.
QJADDP has exactly one output vector: the sum of all 4$-$momenta.
The
following example shows how to use this vector.
\begin{verbatim}
      CALL QJADDP (PSUM, 'ADD-ALL', KRECO, ...)
      ISUM = KPDIR ('ADD-ALL', KRECO)
      CALL HF2 (4711, QP(ISUM), QM(ISUM), 1.)
\end{verbatim}
Other routines may output several vectors; a loop using
KFOLLO (see
\ref{sec-ADE}) must be constructed to access all of them.
\par
Locking an output vector locks all particles contributing to it
(see \ref{sec-QLO}).
You can test whether a track ITK contributes to an output
vector ISUM by using the logical symbol XSAME (ITK, ISUM) (Sec.
\ref{sec-TVATPSO}).
\par
The output vectors of ``QJ'' routines are called
``pseudo$-$particles''. In some routines described
below, these pseudo$-$particles represent an axis
rather than a 3$-$ or 4$-$vector; the momentum value may or may
not be meaningful. For consistency, an energy assuming mass = 0
is calculated in these cases.
\par
In addition, pseudo$-$particles are treated differently than ``real''
particles:
\begin{itemize}
\item A warning is issued if the same name is used for a pseudo$-$particle
and a ``real'' particle.
\item Existing pseudo$-$particles are dropped automatically if the
same name
and the same class is used in another call to a ``QJ'' routine. Thus,
in
\begin{verbatim}
        CALL QJADDP (PSUM, 'ADD-ALL', KRECO, ...)
        CALL QJADDP (PSUM, 'ADD-ALL', KRECO, ...)
        CALL QJADDP (PSUM, 'ADD-ALL', KMONTE, ...)
\end{verbatim}
the output vector of the first call is not available after the second
call. Thus, output vectors from different calls are
never mixed up. Since the third call refers to a different class,
the vector from the second call is not dropped.
Note that you are free to invent new names in
every new call to a ``QJ'' or any other routine.
\end{itemize}
\section{\label{sec-QJEI}Momentum tensor eigenvalues and eigenvectors}
\par
\fbox{CALL QJEIG (EIGVAL, `eigenvector', ICLASS)}
\par
\par
See also QJSPHE in \ref{sec-QJSP} for sphericity value and axis.
 
{\bf Input argument:}
\begin{indentlist}{ 3.00cm}{ 3.25cm}
\indentitem{ICLASS}described in \ref{sec-QJA}.
\end{indentlist}
 
{\bf Results:}
 
\begin{indentlist}{ 3.00cm}{ 3.25cm}
\indentitem{EIGVAL}eigenvalues in descending order
with DIMENSION EIGVAL(3).
\begin{itemize}
\item Sphericity = 1.5 * (1. $-$ EIGVAL(1))
\item Aplanarity = 1.5 * EIGVAL(3)
\item Planarity = EIGVAL(2) $-$ EIGVAL(3)
\end{itemize}
 
\indentitem{'eigenvector'}Three eigenvectors:
\begin{itemize}
\item IMAJOR = KPDIR ('eigenvector', ICLASS)
\item ISEMI = KFOLLO (IMAJOR)
\item IMINOR = KFOLLO (ISEMI)
\end{itemize}
\end{indentlist}
\section{\label{sec-QJEN}Linearized momentum tensor eigenvalues and
eigenvectors}
\par
\fbox{CALL QJTENS (EIGVAL, `eigenvector', ICLASS)}
\par
\par
Same as QJEIG except that a different normalization is used.
The momentum tensor for this calculation is defined as
\begin{equation}
M_{{jk}}={1\over P}\sum _{i}{{p_{{ji}}p_{{ki}}}\over {p_{i}}}
\end{equation}
\begin{equation}
j,k=1,2,3
\end{equation}
\begin{indentlist}{ 3.00cm}{ 3.25cm}
\indentitem{Input arguments and results are as described for QJEIG.}\hspace*{1in}
\end{indentlist}
\section{\label{sec-QJSP}Sphericity}
\par
\fbox{CALL QJSPHE (SPHERI, `spheri$-$axis', ICLASS)}
\par
\par
Calculates sphericity value and sphericity axis.
See also QJEIG in \ref{sec-QJEI} for eigenvalues and eigenvectors
of the momentum tensor.
{\bf Input argument:}
\begin{indentlist}{ 3.00cm}{ 3.25cm}
\indentitem{ICLASS}described in \ref{sec-QJA}.
\end{indentlist}
 
{\bf Results:}
\begin{indentlist}{ 3.00cm}{ 3.25cm}
\indentitem{SPHERI}Sphericity value
\indentitem{'spheri$-$axis'}Sphericity axis.
\end{indentlist}
 
{\bf Error conditions:}
\begin{indentlist}{ 3.00cm}{ 3.25cm}
\indentitem{Zero or one track}SPHERI value 0.; output vector = 0.,0.,0.,0.
\indentitem{Two tracks}SPHERI = 0.; output vector = track vector with
largest p.
\end{indentlist}
\section{\label{sec-QJTH}Thrust}
\par
\fbox{CALL QJTHRU (THRUST, `thrust$-$axis', ICLASS)}
\par
\par
{\bf Input argument:}
\begin{indentlist}{ 3.00cm}{ 3.25cm}
\indentitem{ICLASS}described in \ref{sec-QJA}.
\end{indentlist}
 
{\bf Results:}
\begin{indentlist}{ 3.00cm}{ 3.25cm}
\indentitem{THRUST}Thrust value.
\indentitem{'thrust$-$axis'}Thrust axis.
\end{indentlist}
 
{\bf Error conditions:}
\begin{indentlist}{ 3.00cm}{ 3.25cm}
\indentitem{No track}THRUST value 0.; output vector = 0.,0.,0.,0.
\indentitem{One track}thrust value = 1; output vector = track vector.
\end{indentlist}
{\bf When reading MINIs ( versions $\geq 10$ ) :}
 
  The thrust calculation is very CPU-time consuming. To save time when reading a MINI,
 from MINI 10 onwards (written since December 1994)
 the thrust axis and value are calculated using the 'EF' option in QJOPTR (see \ref{sec-QJORE}). The results
 are stored in a special bank on the MINI.
 When reading these MINIs, the thrust axis and value are not computed from scratch, but
 read directly from this bank.
 
 You don't need to change anything to your program if you want to use
 the 'EF' option.
 If you want to compute the thrust with another option, you MUST give
 the following sequence of calls:
\begin{verbatim}
      XJTHRU=.TRUE.
      CALL QJOPTR ('your reco-option' , 'additional' )
      CALL QJTHRU(THRUST,'thrust-axis',ICLASS)
\end{verbatim}
 
 The logical variable XJTHRU belongs to the QCDE include file of ALPHA, and is set to .FALSE. at the
 beginning of each event.
 
\section{\label{sec-QJFW}Fox$-$Wolfram Moments}
\par
\fbox{CALL QJFOXW(FOXWOL, ICLASS)}
\par
\par
{\bf Input argument:}
\begin{indentlist}{ 3.00cm}{ 3.25cm}
\indentitem{ICLASS}described in \ref{sec-QJA}.
\end{indentlist}
 
{\bf Result:}
\begin{indentlist}{ 3.00cm}{ 3.25cm}
\indentitem{FOXWOL}Fox$-$Wolfram moments H0 $-$ H4;
DIMENSION FOXWOL(5).
\end{indentlist}
 
\section{\label{sec-QJHE}Divide event into two hemispheres}
\par
\fbox{CALL QJHEMI ('same$-$s', `opp$-$s', ICLASS, IVEC, COSCUT)}
\par
\par
{\bf Input arguments:}
\begin{indentlist}{ 3.00cm}{ 3.25cm}
\indentitem{ICLASS}described in \ref{sec-QJA}.
\indentitem{IVEC}Track number of vector which defines the ``hemi''spheres.
\indentitem{COSCUT}The cosine of the opening angle of a cone around
IVEC.
Tracks inside this cone belong to the same side, and all other
ones belong to the opposite side. The word ``hemisphere'' is correct
if COSCUT = 0.
\end{indentlist}
 
{\bf Results:}
\begin{indentlist}{ 3.00cm}{ 3.25cm}
\indentitem{'same$-$s'}The 4$-$momentum sum of tracks on the same
side as IVEC.
\indentitem{'opp$-$s'}
The 4$-$momentum sum of tracks on the side opposite to IVEC.
\end{indentlist}
\par
The two output vectors can be used to assign tracks to one of the
the two hemispheres with the lock algorithm
(\ref{sec-QLO}).
\par In the following example, the event is divided into two
hemispheres according to the thrust axis. Then, each hemisphere is
boosted separately into the rest frame of all contributing tracks.
\begin{verbatim}
      DIMENSION IVECT(2)
C---Thrust axis
      CALL QJTHRU (THRU, 'THRUST', KRECO)
      ITHRU = KPDIR ('THRUST', KRECO)
C---Two hemispheres:
      CALL QJHEMI ('SAME', 'OPPO', KRECO, ITHRU, 0.)
      IVECT(1) = KPDIR ('SAME', KRECO)
      IVECT(2) = KPDIR ('OPPO', KRECO)
C---Lock all tracks in the 'oppo' hemisphere:
      CALL QLOCK (IVECT(2))
C---Loop over both hemispheres:
      DO 10 IHEMI = 1, 2
C---Transform all selected tracks into the rest frame of IVECT(IHEMI):
      CALL QTCLAS (KRECO, IVECT(IHEMI))
C---Now, do the analysis. For example:
C---Plot the thrust in the boosted frame.
      CALL QJTHRU (THRUB, ' ', IVECT(IHEMI))
      CALL QHF1 (4711, THRUB, 1.)
C---QLREV: locked tracks -> unlocked tracks and vice versa.
C---This selects tracks in the hemisphere 'OPPO' for next loop.
      CALL QLREV (KRECO)
   10 CONTINUE
\end{verbatim}
Note that in the above example, two of the maximum six Lorentz frames
are in use. They can be dropped by the statement
CALL QVDROP (' `, IVECT(IHEMI))
inside the loop (see \ref{sec-QVD}).
 
\section{\label{sec-QJME}Missing energy, mass, momentum}
\par
\fbox{CALL QJMISS (PMISS, 'miss$-$vector', ICLASS, ITOTAL)}
\par
\par
{\bf Input arguments:}
\begin{indentlist}{ 4.25cm}{ 4.50cm}
\indentitem{ICLASS}described in \ref{sec-QJA}.
\indentitem{ITOTAL}
= 0: Missing energy, etc. is calculated with respect to
the total energy vector (0.,0.,0.,QELEP).
$>$ 0: Calculation is done with respect to vector ITOTAL.
\end{indentlist}
 
{\bf Results:}
\begin{indentlist}{ 4.25cm}{ 4.50cm}
\indentitem{PMISS}Missing momentum.
\indentitem{'miss$-$vector}vector containing missing
momentum,
mass, and energy.
\end{indentlist}
 
{\bf Error conditions:}
\begin{itemize}
\item Total energy $>$ LEP energy QELEP.
\item Missing momentum $>$ missing energy.
\item In both cases, the
output vector contains energy = PMISS and mass = 0.
\end{itemize}
\section{\label{sec-QJMM}Jet Finding}
\par
\subsection{\label{sec-QJSIM}Scaled Invariant Mass Squared Algorithms}
\par
There are two algorithms available, each with the same 3 variants ( or schemes ).
The first algorithm is known as the JADE algorithm, which defines the invariant mass as:
\par
                                      $ M^{2}= 2E_{1}E_{2}(1-\cos \theta _{12}) $.
\par
The second is known as the DURHAM algorithm and uses a proposal by Dokshitzer for invariant mass,
which is less sensitive to soft gluons:
\par
                                      $ M^{2}= 2(MIN(E_{1},E_{2}))^{2}(1-\cos \theta _{12}) $.
 
\par
Both algorithms combine particles into jets using one of the following combination schemes:
\begin{indentlist}{ 2.50cm}{ 2.75cm}
\indentitem{E   scheme}$ P_{ij} = P_i + P_j $\\
                       $ E_{ij} = E_i + E_j $
\indentitem{P   scheme}$ P_{ij} = P_i + P_j $\\
                       $ E_{ij} = |P_{ij}| $
\indentitem{$E_0$ scheme}$ E_{ij} = E_i + E_j $\\
                       $ P_{ij} = E_{ij}(P_i+P_j)/|P_i+P_j| $
\end{indentlist}
\par
For more details, please read the internal reports ALEPH 90$-149$ (SOFTWR 90$-$015)  and
 ALEPH 91$-$151 ( SOFTWR 91$-$006 ).
 
\par
\subsubsection{\label{sec-QJMMCL}JADE Algorithm with E scheme:}
\par
\fbox{CALL QJMMCL (NJETS, `name', ICLASS, YCUT, EVIS)}
\par
A loop runs over all pairs of tracks and finds the pair which has
the
smallest invariant mass M.
If $
(M/EVIS)^{2}<YCUT
$, these 2 tracks are merged
(i.e., 4$-$momenta added).
\par The loop is then rerun over the new list of tracks which has
lost
2 particles and gained the merged pair. When no remaining pair has
a
low enough mass, the track list contains a set of merged
tracks called jets.
\par
{\bf Input arguments:}
\begin{indentlist}{ 2.50cm}{ 2.75cm}
\indentitem{ICLASS}described in \ref{sec-QJA}.
\indentitem{YCUT}Cut on the scaled invariant mass of 2 tracks.
Pairs of tracks are merged if their scaled invariant mass is smaller
than YCUT.
\indentitem{EVIS}The visible energy of the event.
If EVIS equals 0, the visible energy is computed as the sum of the
input particle energies.
\end{indentlist}
 
{\bf Results:}
\begin{indentlist}{ 2.50cm}{ 2.75cm}
\indentitem{NJETS}is the number of ``jets'' .
\begin{itemize}
\item  $>$ 0 = number of jets found
\item  $-$ 1 = input error with EVIS , or no particles
\item  $-$ 2 = error in one of the particles
\item  $-$ 3 = too many jets found
\item  $-$ 4 = unknown combination scheme requested
\item  $-$ 5 = unknown algorithm requested
\item  $-$ 99 = Not enough BOS workspace to do jet finding
\end{itemize}
\indentitem{'name'}Vectors containing 4$-$momenta of the jets.
\end{indentlist}
\par
EXAMPLE:
\begin{verbatim}
      DIMENSION LISTEJ(300)
      CHARACTER*13 CNAM
C---Select option: charged tracks
      CALL QJOPTR('CH',' ')
C---calculate visible energy from input tracks:
      EVISRE = 0.
      YCUT  = 0.02
      CALL QJMMCL(NJT,'MMCLUS_RE_vis',KRECO,YCUT,EVISRE)
      CNAM = 'MMCLUS_RE_vis'
      WRITE(KUPRNT,*)' # of jets reconstructed ', CNAM, ':', NJT
      IF(NJT.GT.0) THEN
C--- get ALPHA number for first jet found:
        JJ = KPDIR(CNAM,KRECO)
  20    IF(JJ .NE. 0) THEN
C--- get the list of tracks merged into this jet:
          LL = 0
          DO 211 L  = KFCHT, KLCHT
C--- check if this track belongs to this jet:
            IF(.NOT.XSAME(JJ,L)) GOTO 211
            LL = LL + 1
            LISTEJ(LL) = L
  211     CONTINUE
          WRITE(KUPRNT,*) 'Jet # ', J
          WRITE(KUPRNT,*) QX(JJ),QY(JJ),QZ(JJ),QE(JJ)
          WRITE(KUPRNT,*) 'List of tracks merged into this jet:'
          WRITE(KUPRNT,*) (LISTEJ(L),L=1,LL)
C--- get ALPHA number for next jet found:
          JJ = KFOLLO(JJ)
          GOTO 20
        ENDIF
      ENDIF
\end{verbatim}
 
 
\par
\subsubsection{\label{sec-QDMMCL}DURHAM Algorithm with E scheme:}
\par
\fbox{CALL QDMMCL (NJETS, `name', ICLASS, YCUT, EVIS)}
\par
 The arguments and usage are identical to that for QJMMCL above.
\par
 
 
\par
\subsubsection{\label{sec-QGJMMC}Generalised Version : JADE or DURHAM or GENEVA or INVMAS algorithm}
\par
\fbox{CALL QGJMMC (NJETS, `name', ICLASS, YCUT, EVIS,SCHEME,VERSN)}
\par
 The arguments and usage are identical to that for QJMMCL above , except for the extra input arguments :
{\bf Input arguments:}
\begin{indentlist}{ 2.50cm}{ 2.75cm}
\indentitem{SCHEME} CHARACTER*2 : 'E' or 'P' or '$E_0$' (see above)
\indentitem{VERSN } CHARACTER*6 : 'JADE' or 'DURHAM' or 'GENEVA'\\
 or 'INVMAS' to select the corresponding algorithm
\indentitem{YCUT}  - if $>$ 0. :  Cut on the scaled invariant mass of 2 tracks.
Pairs of tracks are merged if their scaled invariant mass is smaller
than YCUT.\\
                   - if $<$ 0. : fixed number of jet asked for, i.e. you can request that the event is clustered until a certain
                   number of jets is found. Example: setting YCUT=-4.  will force the clustering to get 4 jets and only 4 jets as
                   output. 

\end{indentlist}
\par
    Note that 'NORMAL' and 'BETTER' are alternatives for 'JADE' and 'DURHAM' for historical reasons.
\par
    If the 'INVMAS' algorithm is selected, the 'E' scheme is automatically switched on, whatever has been put in the input argument
    SCHEME.
\par

 QJMMCL and QDMMCL both call this routine , which itself calls the ALEPHLIB routine
 FJMMCL , which actually does the jet finding.


\par
\fbox{CALL GETYIJ (MyValues,Nsteps)}                                 
\par
    If you have called QGJMMC with a negative value of YCUT to obtain a fixed number of output jets, this routine returns a vector
    of all Yijs computed, and the number of steps needed to get 4 jets.

 Example:  In your user routine you define a vector:

\begin{verbatim}
      Real*4  MyYvalues(1000)
      Integer Nsteps
      Real EVIS,Y34,Y45,YCUT,EVIS
      Character*2 Scheme
      Character*6 VERSN

   .... then at some stage you call the clustering ...

      call QJOPTR('EF',' ')
      EVIS=0.
      YCUT=-4.
      SCHEME = 'E'
      VERSN  = 'DURHAM'
      call QGJMMC(NJETS,'myjets',KRECO,YCUT,EVIS,SCHEME,VERSN)
      IF (NJETS.LE.0) RETURN  ! Error

   ... and afterwards you call ...

      call GETYIJ(MyYvalues,Nsteps)

   ... this returns a vector of all yijs computed and
       the number of steps needed to get to 4 jets. Therefore

      Y34 = MyYvalues(Nsteps)
      Y45 = MyYvalues(Nsteps-1)

  
\end{verbatim}

 
\subsection{\label{sec-QJMD}Scaled Minimum Distance Algorithm}
\par
\fbox{CALL QJMDCL (NJETS, `name', ICLASS, ALPHA, DELTA, ETA, EVIS)}
\par
\par
A loop runs over all pairs of tracks and finds the pair which has
the
smallest invariant mass M.
If $
(M/EVIS^{\alpha })^{2}<{\sqrt {2(1-\cos 2\delta )}}
$,
these 2 tracks are merged (i.e., 4$-$momenta added).
The loop is then rerun over the new list of tracks which has lost
2 particles and gained the merged pair. When no remaining pair has
a
low enough mass, the track list contains a set of merged
tracks. If these tracks have energies bigger than $
2\eta Evis
$,
they are called jets.
\par
{\bf Input arguments:}
\begin{indentlist}{ 2.50cm}{ 2.75cm}
\indentitem{ICLASS}described in \ref{sec-QJA}.
\indentitem{ALPHA}Weight of track energies and Evis, in the calculation
of
the scaled mass.
Pairs of tracks are merged if their scaled mass is smaller
than $
{\sqrt {2(1-\cos 2\delta )}}
$.
\indentitem{DELTA}Half opening angle cut in degrees.
\indentitem{ETA}Cut on jet energies (fraction of $
2Evis
$);
only jets with energies $
>2\eta Evis
$ are kept.
\indentitem{EVIS}The visible energy of the event;
if EVIS equals 0 , the visible energy is computed as the sum of the
input particle energies.
\end{indentlist}
 
{\bf Results:}
\begin{indentlist}{ 2.50cm}{ 2.75cm}
\indentitem{NJETS }is the number of ``jets''.
\indentitem{'name'}Vectors containing 4$-$momenta of the jets.
\end{indentlist}
 
\subsection{\label{sec-QJLU}JETSET algorithm LUCLUS from LUND}
\par
\fbox{CALL QJLUCL (NJETS, `name', ICLASS, MINCLU, DMAX1, DMAX2,MULSYM, TGEN,
DMIN)}
\par
\par
{\bf Input arguments:}
\begin{indentlist}{ 2.50cm}{ 2.75cm}
\indentitem{ICLASS}described in \ref{sec-QJA}.
\indentitem{MINCLU}Minimum number of clusters to be reconstructed.
(if $<$0, work space momenta are used as a start)
(usually=1)
\indentitem{DMAX1}Max. distance to form starting clusters (usually=0.25GeV)
\indentitem{DMAX2}Max. distance to join 2 clusters (usually=2.5 GeV)
\indentitem{MULSYM}
\begin{itemize}
\item = 1 for symmetric distance criterion (usual)
\item = 2 for multicity distance criterion
\end{itemize}
\end{indentlist}
 
{\bf Results:}
\begin{indentlist}{ 2.50cm}{ 2.75cm}
\indentitem{NJETS} is the number of ``jets''
\begin{itemize}
\item = $-$1 if not enough particles
\item = $-$2 if not enough working space (KTBOMX)
\end{itemize}
\indentitem{TGEN}Generalized thrust
\indentitem{DMIN}Minimum distance between 2 jets
\begin{itemize}
\item = 0  when only 1 jet
\item = $-$1 , $-$2 as for NJET
\end{itemize}
\indentitem{'name'}Vectors containing 4$-$momenta of the jets.
\end{indentlist}
 
\subsection{\label{sec-QJPT}PTCLUS: Jet-finding algorithm}
\par
\fbox{CALL QJPTCL (NJETS,'name',ICLASS,NJTLIM,YJTLIM,EVIS)}
\par
\par
The PTCLUS jet-finding algorithm is described in the note ALEPH 89 $-$ 150 (PHYSIC 89-060) by M. Scarr and I. TenHave.
\par
{\bf Input arguments:}
\begin{indentlist}{ 2.50cm}{ 2.75cm}
\indentitem{ICLASS}described in \ref{sec-QJA}.
\indentitem{NJLITM}maximum number of jets to search for; if NJLITM=0,
the algorithm finds the number of jets using YJTLIM.
\indentitem{YJLITM}maximum allowed distance between two clusters
(in M$^2$ / EVIS$^2$); 0.02 is a typical value.
\indentitem{EVIS}visible energy.  If EVIS=0, the visible energy is
calculated.
\end{indentlist}
 
{\bf Results:}
\begin{indentlist}{ 2.50cm}{ 2.75cm}
\indentitem{NJETS} is the number of ``jets''. ($-$1 if algorithm fails)
\indentitem{TGEN}Generalized thrust
\indentitem{'name'}Vectors containing 4$-$momenta of the jets.
\end{indentlist}

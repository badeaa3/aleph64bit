\chapter{\label{sec-EF}Energy Flow}
\par
Historically, three energy flow packages have been used in ALEPH:  the mask
algorithm of M-N. Minard
and M. Pepe-Altarelli,
the PCPA$-$based energy flow of A. Bonissent, and
the ENFLW package
of P. Janot.
Since ALPHA 115 (May 1993),
the ENFLW package is fully integrated in ALPHA;                             
all features of the other two packages are not maintained any more.
Therefore, users are strongly advised to use the ENFLW energy flow.
\par
\section{\label{sec-EFLWJ}ENFLW Energy Flow}
\par
  The ENFLW algorithm is described in the Section 10 of the paper:
 
   `` Performances of the ALEPH detector at LEP ''
 
  Nucl. Inst. and Meth. {\bf A 360} (1995), pp. 481$-$506 .
 
\par
To use the ENFLW
energy flow analysis, the EFLW card must be given in the ALPHA card
file.
If the EFLW card is present, the EFT section of ALPHA is filled with
selected charged tracks and neutral ECAL and HCAL clusters.
These objects can be accessed
with DO loops
(KFEFT, KLEFT, KNEFT -- see \ref{sec-altrack})
or with the particle name `EFLW' using
the functions KPDIR and KFOLLO (described in
\ref{sec-AD}).


The charged tracks that appear in the EFT section are copies of
standard ALPHA charged tracks from the CHT section.
Therefore, if a charged track in the CHT section is locked (using
QLTRK or QLOCK), the corresponding track in the EFT section will be
locked also (and vice versa).

All statement functions providing information about charged tracks can be used
directly with charged tracks in the EFT section .
 
\subsection{\label{sec-EFLWS}Access to ENFLW informations}

The following statement functions may be used to access
additional information on EFLW objects:
 
\begin{indentlist}{ 3.00cm}{ 3.25cm}
\indentitem{XEFO (I)}{\bf .TRUE.} if energy flow 
data are available for ``track'' I


\newpage

\indentitem{KEFOTY (I)}Type of energy flow objects (exclusive list, no double counting):
\begin{itemize}
\item 0 = Charged Track (Pion assumed, not identified either e or mu) 
\item 1 = Electron
\item 2 = Muon
\item 3 = Track from a standard V0 (either $\Lambda, K^{0}_{s}$ or $\gamma$ conversion) from the YV0V package 
\item 4 = Electromagnetic ($\gamma$ or $\pi^{0}$)
\item 5 = ECAL hadron/residual
\item 6 = HCAL element
\item 7 = LCAL element (No Particle Identification available for LCAL)
\item 8 = SICAL element (No Particle Identification available for SICAL)
         
\end{itemize}
\indentitem{KEFOLE (I)}PECO number of associated ECAL object
\indentitem{KEFOLT (I)}FRFT number of associated charged track
\indentitem{KEFOLH (I)}PHCO number of associated HCAL object
\indentitem{KEFOLC (I)}PCOB number of associated calorimeter object
\indentitem{KEFOLJ (I)}EJET number of associated jet
\end{indentlist}
\par

\subsection{\label{sec-EFLWT}Event topology routines and ENFLW}
To use the event topology routines described in
Chapter \ref{sec-QJ} with these energy-flow objects,
use option {\bf 'EF'} with subroutine QJOPTR
(see~\ref{sec-QJOP}):
\begin{verbatim}
      CALL QJOPTR('EF',' ')
\end{verbatim}
 
\noindent {\bf Example:}
 
The following code calculates the total energy energy of an
event and finds the thrust using energy flow objects.
 
\begin{verbatim}
      E=0.
      DO 10 I = KFEFT, KLEFT
         E=E + QE(I)
   10 CONTINUE
C---    Find thrust
      CALL QJOPTR('EF', ' ')
      CALL QJTHRU(THRU, 'THRU', KRECO)
\end{verbatim}
\par
Jets based on energy flow objects using QJMMCL with YCUT = 0.003
(see Sec.~\ref{sec-QJSIM})
are
stored in the EJET bank.
If the EFLJ card is used instead of the EFLW card, the EFT section will
be filled as described above, and these jets will be stored in the JET
section.
The jets may be accessed with DO loops (KFJET, KLJET, KNJET)
or with the particle name 'EJET' using the functions KPDIR and KFOLLO.
The energy flow objects making up these jets can be found with XSAME
as described in Sec.~\ref{sec-TVATPSO}.
To save time, these jets may be used as input for jet-finding with a
higher YCUT (see~\ref{sec-QJSIM})
by calling QJOPTR with the option EJ:
\begin{verbatim}
     CALL QJOPTR('EJ',' ').
\end{verbatim}
XSAME may be used to find the original energy flow objects
(in the EFT section) making up the final jets.
 

\subsection{\label{sec-EFLWR}Removing of SiCAL clusters in ENFLW}
 
 A special data card:  NOSC  is foreseen to remove  the SiCAL clusters from ENFLW  objects.
    This works on POT/DST/MINI.
\par
      However, be careful:
   when writing  MINIs from DST/POT with NOSC,
      the SiCAL clusters are NOT   saved on the MINIs and will not be recovered when reading these MINIs
                                    even without NOSC. This is OK
      for private productions; the official MINIs are
         produced with the SICAL clusters in energy flow objects.

\subsection{\label{sec-EFLSICL}Cleanup of bad SiCAL clusters in ENFLW}

Since ALPHA 124.09 (May 1999) all bad SiCAL clusters are removed from the energy flow objects. This is done
by calling the ALEPHLIB subroutine SICLID. This works for all POTs, but works only  for DSTs and MINIs processed with
 JULIA version 309 and after.

This cleanup may be inhibited by a special data card, NSID. 



\subsection{\label{sec-EFLWMA}Matching ENFLW informations and true information}

The code providing this information is available since ALPHA 122.30 (June 1997).

For charged tracks, the matching between ENFLW objects and Monte Carlo true information
(and {\it vice-versa}) is done by the standard KMTCH function
(\ref{sec-AX} on p.~\pageref{sec-AX}).

For ECAL objects, it is done using the PEMH, PECO and PCOB banks.  
For HCAL objects, it is done using the PHMH, PHCO and PCOB banks for datasets generated with JULIA version 285.1 and after
(March 1996 and after); for previous datasets a tight angular matching (100 mrad) is used as a substitute when PHMH is not
available. For ECAL hadronic residuals (KEFOTY = 5) and SICAL clusters (KEFOTY = 8) the matching is not fully efficient, since
the FKIN/PECO information is not stored for SICAL clusters.


The matching information is provided through the following functions:

\begin{indentlist}{ 3.00cm}{ 3.25cm}
\indentitem{KNTRU (ITK)} Number of matches of ALPHA ``track" ITK                    

                         The matching is ENFLW $\rightarrow$ Monte Carlo if ITK is between KFEFT and KLEFT
  
                         The matching is Monte Carlo $\rightarrow$  ENFLW if ITK is between KFMCT and KLMCT
 
                         As for charged tracks, any ENFLW object may have more than one match, therefore  
                         additional information is provided through the following functions:

\indentitem{KTRU (ITK,I)} Ith match to  ALPHA ``track" ITK (I=1,.. KNTRU(ITK))

\indentitem{QSTRU(ITK,I)} Matching quantity for the Ith match to  ALPHA ``track" ITK
\begin{itemize}
\item if QSTRU $<-1.0$:  (-QSTRU) is the number of shared VDET + ITC + TPC hits
\item if $-1.0 < $ QSTRU $<0.$ (-QSTRU) is the matching angle (in radians) 
                for HCAL/SICAL
\item if QSTRU $>0.$ QSTRU is the shared energy (from PEMH or PHMH banks)   
\end{itemize}

\end{indentlist}


\par
\section{\label{sec-EFLWM}Mask Energy Flow}
\par
This algorithm is described in the report ALEPH 89$-110$ (PHYSIC 89$-$043).
Its results are not written any more in ALEPH datasets.

\par
\section{\label{sec-EFLWP}PCPA-based Energy Flow}
\par
This algorithm is described in the report ALEPH 92$-055$ (PHYSIC 92$-$048).
\par
The PCPA-based energy flow uses neutral objects derived from the
PCPA bank in addition to selected charged tracks. The logical
function XFRIQF(ITK) may be used to test whether a track has been
included for the PCPA energy-flow analysis.
The PCPA neutral objects are stored in the NET section by default.
(Filling of the NEOB section may be disabled by including the card
NOPC in
the ALPHA card file.)
These objects can be accessed with DO loops
(KFNET, KLNET, KNNET -- see \ref{sec-altrack})
or with the particle name `NEOB' using
the functions KPDIR and KFOLLO (described in
\ref{sec-AD}).
 
\par
 
To use the event topology routines described in
Chapter \ref{sec-QJ} with PCPA-based energy flow  (\ie,
selected charged tracks plus PCPA neutral objects),
use option {\bf 'PC'} with subroutine QJOPTR
(see~\ref{sec-QJOP}):
\begin{verbatim}
      CALL QJOPTR('PC',' ').
\end{verbatim}
To use only NEOB objects:
\begin{verbatim}
      CALL QJOPTR('NO','NEOB').
\end{verbatim}
 
The following statement functions may be used to access
additional information on NEOB objects:
 
\begin{indentlist}{ 3.00cm}{ 3.25cm}
\indentitem{XPCQ (I)}{\bf .TRUE.} if PCQA data are available for ``track'' I
\indentitem{KPCQNA (I)}NAture of neutral object (see Sec.~\ref{sec-EFLWP})
\begin{itemize}
\item 1 Isolated gamma
\item 2 Gamma from multi-gamma neutral cluster
\item 3 Gamma from identified $\pi^0$
\item 4 Gamma from electron bremsstrahlung
\item 5 Gamma from electromagnetic charged cluster
\item 10 Unresolved gamma-gamma
\item 12 Residual electromagnetic energy from neutral cluster
\item 13 Residual electromagnetic energy from charged cluster
\item 17 Neutral hadron
\item 18 Residual hadronic energy from neutral cal object
\item 19 Residual hadronic energy from
charged cal object with no HCAL component
\item 20 Residual hadronic energy from
charged cal object with HCAL component
\item 21 contribution from an ECAL cluster for
which EBNEUT was in error
\item 22 contribution from an LCAL object
 
\end{itemize}
\end{indentlist}
 
{\bf Example:}
 
The following code calculates the total energy energy of an
event and finds the sphericity using PCPA-based energy flow.
 
\begin{verbatim}
C---    First add up neutral energy
    E = 0.
    DO 10 I = KFNET, KLNET
    E=E+QE(I)
10  CONTINUE
C---    Add energies of selected tracks
    DO 20 I = KFCHT, KLCHT
    IF(XFRIQF(I)) E = E + QE(I)
20  CONTINUE
C---    Find sphericity
    CALL QJOPTR('PC', ' ')
    CALL QJSPHE(SPHE, 'SPHE', KRECO)
\end{verbatim}
